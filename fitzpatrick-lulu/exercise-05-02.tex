\input{preamble}

\FBOX{
5-2. A particle of mass $m$ moves freely in one dimension between
impenetrable walls located at $x=0$ and $a$.
Its initial wavefunction is
\begin{equation*}
\psi(x,0)=\sqrt{2/a}\sin(3\pi x/a).
\end{equation*}

What is the subsequent time evolution of the wavefunction?
Suppose that the initial wavefunction is
\begin{equation*}
\psi(x,0)=\sqrt{1/a}\sin(\pi x/a)[1+2\cos(\pi x/a)].
\end{equation*}

What now is the subsequent time evolution?
Calculate the probability of finding the particle
between 0 and $a/2$ as a function of time in each case.
}

Here is equation (5.11).
\begin{equation*}
\psi_n(x)=\sqrt{\frac{2}{a}}\sin\left(n\pi\frac{x}{a}\right)
\tag{5.11}
\end{equation*}

Hence for $\sin(3\pi x/a)$ we have $n=3$.
Then by equation (5.12) the time evolution is
\begin{equation*}
\psi(x,t)=\sqrt{\frac{2}{a}}\sin\left(\frac{3\pi x}{a}\right)
\exp\left(-\frac{iE_3t}{\hbar}\right)
\end{equation*}

and from equation (5.9)
\begin{equation*}
E_3=\frac{3^2\pi^2\hbar^2}{2ma^2}
\end{equation*}

By the identity $2\sin(x)\cos(x)=\sin(2x)$ we have
\begin{equation*}
\sin(\pi x/a)[1+2\cos(\pi x/a)]
=\sin(\pi x/a)+\sin(2\pi x/a)
\end{equation*}

Hence for the second wavefunction
\begin{equation*}
\psi(x,t)=\frac{1}{\sqrt a}\left(
\sin\left(\frac{\pi x}{a}\right)
\exp\left(-\frac{iE_1t}{\hbar}\right)
+\sin\left(\frac{2\pi x}{a}\right)
\exp\left(-\frac{iE_2t}{\hbar}\right)
\right)
\end{equation*}

For the first wavefunction, the probability is
\begin{equation*}
\int_0^{a/2}\psi^*\psi\,dx=\frac{1}{2}
\end{equation*}

For the second wavefunction
\begin{equation*}
\int_0^{a/2}\psi^*\psi\,dx=\frac{4}{3\pi}
\cos\left(\frac{(E_1-E_2)t}{\hbar}\right)+\frac{1}{2}
\end{equation*}

Noting that
\begin{equation*}
\frac{4}{3\pi}\approx0.42
\end{equation*}

the probability oscillates between 0.08 and 0.92 with the
frequency determined by $E_1-E_2$.

\end{document}
