\input{preamble}

\FBOX{
4-8. An operator $A$, corresponding to a physical quantity $\alpha$,
has two normalized eigenfunctions $\psi_1(x)$ and $\psi_2(x)$,
with eigenvalues $a_1$ and $a_2$.
An operator $B$, corresponding to another physical quantity $\beta$,
has normalized eigenfunctions $\phi_1(x)$ and $\phi_2(x)$,
with eigenvalues $b_1$ and $b_2$.
The eigenfunctions are related via
\begin{align*}
\psi_1&=(2\phi_1+3\phi_2)/\sqrt{13},
\\
\psi_2&=(3\phi_1-2\phi_2)/\sqrt{13}.
\end{align*}

$\alpha$ is measured and the value $a_1$ is obtained.
If $\beta$ is then measured and then $\alpha$ again, show
that the probability of obtaining $a_1$ a second time is $97/169$.
}

In matrix form we have
\begin{equation*}
\begin{pmatrix}\psi_1\\\psi_2\end{pmatrix}
=\frac{1}{\sqrt{13}}
\begin{pmatrix}2&3\\3&-2\end{pmatrix}
\begin{pmatrix}\phi_1\\\phi_2\end{pmatrix}
\end{equation*}

Noting that
\begin{equation*}
\begin{pmatrix}2&3\\3&-2\end{pmatrix}^{-1}
=\frac{1}{13}\begin{pmatrix}2&3\\3&-2\end{pmatrix}
\end{equation*}

we have for $\phi$ in terms of $\psi$
\begin{equation*}
\begin{pmatrix}\phi_1\\\phi_2\end{pmatrix}
=\frac{1}{\sqrt{13}}
\begin{pmatrix}2&3\\3&-2\end{pmatrix}
\begin{pmatrix}\psi_1\\\psi_2\end{pmatrix}
\end{equation*}

After the first measurement $\alpha=a_1$ the system is in state $\psi_1$.
Noting that
\begin{equation*}
\psi_1=\frac{2}{\sqrt{13}}\phi_1+\frac{3}{\sqrt{13}}\phi_2
\end{equation*}

the probabilities for the measurement of $\beta$ are
\begin{align*}
P(\beta=b_1)&=\left(\frac{2}{\sqrt{13}}\right)^2=\frac{4}{13}
\\[1ex]
P(\beta=b_2)&=\left(\frac{3}{\sqrt{13}}\right)^2=\frac{9}{13}
\end{align*}

Suppose $\beta=b_1$ is measured.
Then the system is now in state $\phi_1$ and
\begin{equation*}
\phi_1=\frac{2}{\sqrt{13}}\psi_1+\frac{3}{\sqrt{13}}\psi_2
\end{equation*}

Hence the probabilities for the second measurement of $\alpha$ are
\begin{align*}
P(\alpha=a_1\cap\beta=b_1)
&=\left(\frac{2}{\sqrt{13}}\right)^2P(\beta=b_1)
=\frac{16}{169}
\\[1ex]
P(\alpha=a_2\cap\beta=b_1)
&=\left(\frac{3}{\sqrt{13}}\right)^2P(\beta=b_1)
=\frac{36}{169}
\end{align*}

Suppose $\beta=b_2$ is measured.
Then the system is now in state $\phi_2$ and
\begin{equation*}
\phi_2=\frac{3}{\sqrt{13}}\psi_1-\frac{2}{\sqrt{13}}\psi_2
\end{equation*}

Hence the probabilities for the second measurement of $\alpha$ are
\begin{align*}
P(\alpha=a_1\cap\beta=b_2)
&=\left(\frac{3}{\sqrt{13}}\right)^2P(\beta=b_2)
=\frac{81}{169}
\\[1ex]
P(\alpha=a_2\cap\beta=b_2)
&=\left(\frac{-2}{\sqrt{13}}\right)^2P(\beta=b_2)
=\frac{36}{169}
\end{align*}

By total probability
\begin{equation*}
P(\alpha=a_1)
=P(\alpha=a_1\cap\beta=b_1)+P(\alpha=a_1\cap\beta=b_2)
=\frac{16}{169}+\frac{81}{169}=\frac{97}{169}
\end{equation*}

\end{document}
