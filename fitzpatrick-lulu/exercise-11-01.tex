\input{preamble}

\FBOX{
11-1. An electron in a hydrogen atom occupies the combined spin
and position state
\begin{equation*}
R_{2,1}\left(\sqrt{1/3}Y_{1,0}\chi_+
+\sqrt{2/3}Y_{1,1}\chi_-\right)
\end{equation*}

(a) What values would a measurement of $L^2$ yield,
and with what probabilities?

\bigskip
(b) Same for $L_z$.

\bigskip
(c) Same for $S^2$.

\bigskip
(d) Same for $S_z$.

\bigskip
(e) Same for $J^2$.

\bigskip
(f) Same for $J_z$.

\bigskip
(g) What is the probability density for finding the electron at
$r$, $\theta$, $\phi$?

\bigskip
(h) What is the probability density for finding the electron in the
spin up state (with respect to the $z$-axis) at radius $r$?
}

(a) By $L^2Y_{l,m}=l(l+1)\hbar^2Y_{l,m}$ we have
\begin{align*}
\Pr\left(L^2=2\hbar^2\cap m=0\right)
&=\left|\langle Y_{1,0}|\sqrt{1/3}Y_{1,0}+\sqrt{2/3}Y_{1,1}\rangle\right|^2=\tfrac{1}{3}
\\[1ex]
\Pr\left(L^2=2\hbar^2\cap m=1\right)
&=\left|\langle Y_{1,1}|\sqrt{1/3}Y_{1,0}+\sqrt{2/3}Y_{1,1}\rangle\right|^2=\tfrac{2}{3}
\end{align*}

By total probability
\begin{equation*}
\Pr\left(L^2=2\hbar^2\right)=\tfrac{1}{3}+\tfrac{2}{3}=1
\end{equation*}

(b) By $L_zY_{l,m}=m\hbar Y_{l,m}$ we have
\begin{align*}
\Pr\left(L_z=0\right)=\left|\langle Y_{1,0}|\sqrt{1/3}Y_{1,0}+\sqrt{2/3}Y_{1,1}\rangle\right|^2=\tfrac{1}{3}
\\[1ex]
\Pr\left(L_z=\hbar\right)=\left|\langle Y_{1,1}|\sqrt{1/3}Y_{1,0}+\sqrt{2/3}Y_{1,1}\rangle\right|^2=\tfrac{2}{3}
\end{align*}

(c) Recall
\begin{equation*}
\chi_\pm\equiv\chi_{s=\tfrac{1}{2},m_s=\pm\tfrac{1}{2}}
\end{equation*}

and
\begin{equation*}
S^2\chi_\pm=s(s+1)\hbar^2=\tfrac{3}{4}\hbar^2\chi_\pm
\end{equation*}

Hence
\begin{align*}
\Pr\left(S^2=\tfrac{3}{4}\hbar^2\cap m_s=+\tfrac{1}{2}\right)
&=\left|\langle\chi_+|\sqrt{1/3}\chi_++\sqrt{2/3}\chi_-\rangle\right|^2=\tfrac{1}{3}
\\[1ex]
\Pr\left(S^2=\tfrac{3}{4}\hbar^2\cap m_s=-\tfrac{1}{2}\right)
&=\left|\langle\chi_-|\sqrt{1/3}\chi_++\sqrt{2/3}\chi_-\rangle\right|^2=\tfrac{2}{3}
\end{align*}

By total probability
\begin{equation*}
\Pr\left(S^2=\tfrac{3}{4}\hbar^2\right)=\tfrac{1}{3}+\tfrac{2}{3}=1
\end{equation*}

(d) By
\begin{equation*}
S_z\chi_\pm=\pm\tfrac{1}{2}\hbar\chi_\pm
\end{equation*}

we have
\begin{align*}
\Pr\left(S_z=+\tfrac{1}{2}\hbar\right)
&=\left|\langle\chi_+|\sqrt{1/3}\chi_++\sqrt{2/3}\chi_-\rangle\right|^2=\tfrac{1}{3}
\\[1ex]
\Pr\left(S_z=-\tfrac{1}{2}\hbar\right)
&=\left|\langle\chi_-|\sqrt{1/3}\chi_++\sqrt{2/3}\chi_-\rangle\right|^2=\tfrac{2}{3}
\end{align*}

(e)

\bigskip
(f)

\bigskip
(g)
\begin{equation*}
f(r,\theta,\phi)=|R_{21}|^2\left(\tfrac{1}{3}|Y_{1,0}|^2+\tfrac{2}{3}|Y_{1,1}|^2\right)
\end{equation*}

(h) The joint probability density function is
\begin{equation*}
f(r,S_z)=\tfrac{1}{3}|R_{21}|^2I\left(S_z=\tfrac{1}{2}\hbar\right)
+\tfrac{2}{3}|R_{21}|^2I\left(S_z=-\tfrac{1}{2}\hbar\right)
\end{equation*}

where $I$ is the indicator function.

\bigskip
Hence the probability density function for finding the electron in the
spin up state at radius $r$ is
\begin{equation*}
f\left(r,\tfrac{1}{2}\hbar\right)=\tfrac{1}{3}|R_{21}|^2
\end{equation*}

\end{document}
