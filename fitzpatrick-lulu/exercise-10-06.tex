\input{preamble}

\FBOX{
10-6. An electron is at rest in an oscillating magnetic field
\begin{equation*}
\mathbf B=B_0\cos(\omega t)\,\mathbf e_z,
\end{equation*}

where $B_0$ and $\omega$ are real positive constants.

\bigskip
(a) Find the Hamiltonian of the system.

\bigskip
(b) If the electron starts in the spin-up state with respect to the
$x$-axis, determine the spinor
$\chi(t)$ which represents the state of the system in the Pauli
representation at all subsequent times.

\bigskip
(c) Find the probability that a measurement of $S_x$ yields the result
$-\hbar/2$ as a function of time.

\bigskip
(d) What is the minimum value of $B_0$ required to force a complete
flip in $S_x$?
}

(a) The Hamiltonian is
\begin{equation*}
H=\frac{geB_0\cos(\omega t+\phi)}{2m_e}\,S_z
\end{equation*}

(b) Let
\begin{equation*}
\chi(t)=\begin{pmatrix}c_+(t)\\c_-(t)\end{pmatrix}
\end{equation*}

By the Schrodinger equation we have
\begin{equation*}
i\hbar\frac{\partial}{\partial t}\chi=H\chi
\end{equation*}

Hence
\begin{equation*}
i\hbar
\begin{pmatrix}\frac{\partial}{\partial t}c_+(t)\\[1ex]\frac{\partial}{\partial t}c_-(t)\end{pmatrix}
=\frac{geB_0\cos(\omega t+\phi)}{2m_e}
\frac{\hbar}{2}\begin{pmatrix}1&0\\0&-1\end{pmatrix}\begin{pmatrix}c_+(t)\\c_-(t)\end{pmatrix}
\end{equation*}

The solution is
\begin{align*}
c_+(t)&=C\exp\left(-\frac{igeB_0\sin(\omega t+\phi)}{4m_e\omega}\right)
\end{align*}

and
\begin{align*}
c_-(t)&=C\exp\left(\frac{igeB_0\sin(\omega t+\phi)}{4m_e\omega}\right)
\end{align*}

The electron starts out in the spin-up state with respect
to the $x$-axis hence
\begin{equation*}
\chi(0)=\begin{pmatrix}c_+(0)\\c_-(0)\end{pmatrix}
=\frac{1}{\sqrt2}\begin{pmatrix}1\\1\end{pmatrix}
\end{equation*}

Hence $C=1/\sqrt2$ and $\phi=0$.

\bigskip
(c) From the eigenstate
\begin{equation*}
x_-=\frac{1}{\sqrt2}\begin{pmatrix}1\\-1\end{pmatrix}
\end{equation*}

we have
\begin{equation*}
\Pr\left(S_x=-\tfrac{\hbar}{2}\right)=\left|\langle x_-|\chi\rangle\right|^2
=\frac{1}{2}-\frac{1}{2}\cos\left(\frac{geB_0\sin(\omega t)}{2m_e\omega}\right)
\end{equation*}

(d) The expectation for $S_x$ is
\begin{equation*}
\langle S_x\rangle=\langle\chi|S_x|\chi\rangle
=\frac{\hbar}{2}\cos\left(\frac{geB_0\sin(\omega t)}{2m_e\omega}\right)
\end{equation*}

As a result of $-1\le\sin(\omega t)\le1$, the cosine is always positive for
\begin{equation*}
\frac{geB_0}{2m_e\omega}\le\frac{\pi}{2}
\end{equation*}

Hence to change the sign of $\langle S_x\rangle$ requires
\begin{equation*}
B_0>\frac{\pi m_e\omega}{ge}
\end{equation*}

\end{document}
