\input{preamble}

\FBOX{
8-3. Consider a beam of particles with $l=1$.
A measurement of $L_x$ yields the result $\hbar$.
What values will be obtained by a subsequent measurement of $L_z$,
and with what probabilities?
Repeat the calculation for the cases in which the measurement of $L_x$
yields the results $0$ and $-\hbar$.
}

After measuring $L_x=\hbar$ the state is $\psi_{1,1}$ where (see exercise 8-2)
\begin{equation*}
\psi_{1,1}=\frac{1}{2}Y_{1,1}+\frac{1}{\sqrt2}Y_{1,0}+\frac{1}{2}Y_{1,-1}
\end{equation*}

Hence the probabilities for $L_z$ are
\begin{align*}
\Pr(L_z=1)&=\left|\langle Y_{1,1}|\psi_{1,1}\rangle\right|^2=\tfrac{1}{4}
\\[1ex]
\Pr(L_z=0)&=\left|\langle Y_{1,0}|\psi_{1,1}\rangle\right|^2=\tfrac{1}{2}
\\[1ex]
\Pr(L_z=-1)&=\left|\langle Y_{1,-1}|\psi_{1,1}\rangle\right|^2=\tfrac{1}{4}
\end{align*}

After measuring $L_x=0$ the state is $\psi_{1,0}$ where
\begin{equation*}
\psi_{1,0}=-\frac{1}{\sqrt2}Y_{1,1}+\frac{1}{\sqrt2}Y_{1,-1}
\end{equation*}

Hence the probabilities for $L_z$ are
\begin{align*}
\Pr(L_z=1)&=\left|\langle Y_{1,1}|\psi_{1,0}\rangle\right|^2=\tfrac{1}{2}
\\[1ex]
\Pr(L_z=-1)&=\left|\langle Y_{1,-1}|\psi_{1,0}\rangle\right|^2=\tfrac{1}{2}
\end{align*}

After measuring $L_x=-\hbar$ the state is $\psi_{1,-1}$ where
\begin{equation*}
\psi_{1,-1}=-\frac{1}{2}Y_{1,1}+\frac{1}{\sqrt2}Y_{1,0}-\frac{1}{2}Y_{1,-1}
\end{equation*}

Hence the probabilities for $L_z$ are
\begin{align*}
\Pr(L_z=1)&=\left|\langle Y_{1,1}|\psi_{1,-1}\rangle\right|^2=\tfrac{1}{4}
\\[1ex]
\Pr(L_z=0)&=\left|\langle Y_{1,0}|\psi_{1,-1}\rangle\right|^2=\tfrac{1}{2}
\\[1ex]
\Pr(L_z=-1)&=\left|\langle Y_{1,-1}|\psi_{1,-1}\rangle\right|^2=\tfrac{1}{4}
\end{align*}

Note that we had to know $l=1$.
We could also observe $L_x=\hbar$ for $l=2$, etc.

\end{document}
