\input{preamble}

\FBOX{
10-4. An electron is in the spin-state
\begin{equation*}
\chi=A\begin{pmatrix}1-2i\\2\end{pmatrix}
\end{equation*}

in the Pauli representation.
Determine the constant $A$ by normalizing $\chi$.
If a measurement of $S_z$ is made, what values will be obtained,
and with what probabilities?
What is the expectation value of $S_z$?
Repeat the above calculations for $S_x$ and $S_y$.
}

Noting that
\begin{equation*}
|1-2i|^2+|2|^2=9
\end{equation*}

the normalization constant is
\begin{equation*}
A=\tfrac{1}{3}
\end{equation*}

hence
\begin{equation*}
\chi=\begin{pmatrix}\frac{1}{3}-\frac{2}{3}i\\[1ex]\frac{2}{3}\end{pmatrix}
\end{equation*}

These are the eigenstates for spin.
\begin{align*}
|x_+\rangle&=\tfrac{1}{\sqrt2}(1,1) &
|y_+\rangle&=\tfrac{1}{\sqrt2}(1,i) &
|z_+\rangle&=(1,0)
\\
|x_-\rangle&=\tfrac{1}{\sqrt2}(1,-1) &
|y_-\rangle&=\tfrac{1}{\sqrt2}(1,-i) &
|z_-\rangle&=(0,1)
\end{align*}

For probabilities in the $z$ direction we have
\begin{equation*}
\Pr\left(S_z=+\tfrac{\hbar}{2}\right)=|\langle z_+|\chi\rangle|^2=\tfrac{5}{9},\quad
\Pr\left(S_z=-\tfrac{\hbar}{2}\right)=|\langle z_-|\chi\rangle|^2=\tfrac{4}{9}
\end{equation*}

For the expectation of $S_z$ we have
\begin{equation*}
\langle z\rangle=\langle\chi|\sigma_z|\chi\rangle=\tfrac{1}{9},\quad
\langle S_z\rangle=\tfrac{\hbar}{2}\langle z\rangle=\tfrac{1}{18}\hbar
\end{equation*}

For probabilities in the $x$ direction we have
\begin{equation*}
\Pr\left(S_x=+\tfrac{\hbar}{2}\right)=|\langle x_+|\chi\rangle|^2=\tfrac{13}{18},\quad
\Pr\left(S_x=-\tfrac{\hbar}{2}\right)=|\langle x_-|\chi\rangle|^2=\tfrac{5}{18}
\end{equation*}

For the expectation of $S_x$ we have
\begin{equation*}
\langle x\rangle=\langle\chi|\sigma_x|\chi\rangle=\tfrac{4}{9},\quad
\langle S_x\rangle=\tfrac{\hbar}{2}\langle x\rangle=\tfrac{2}{9}\hbar
\end{equation*}

For probabilities in the $y$ direction we have
\begin{equation*}
\Pr\left(S_y=+\tfrac{\hbar}{2}\right)=|\langle y_+|\chi\rangle|^2=\tfrac{17}{18},\quad
\Pr\left(S_y=-\tfrac{\hbar}{2}\right)=|\langle y_-|\chi\rangle|^2=\tfrac{1}{18}
\end{equation*}

For the expectation of $S_y$ we have
\begin{equation*}
\langle y\rangle=\langle\chi|\sigma_y|\chi\rangle=\tfrac{8}{9},\quad
\langle S_y\rangle=\tfrac{\hbar}{2}\langle y\rangle=\tfrac{4}{9}\hbar
\end{equation*}

Adding the following to the exercise.

\bigskip
The spin polarization vector is
\begin{equation*}
\mathbf u=\frac{2}{\hbar}
\begin{pmatrix}
\langle S_x\rangle
\\[1ex]
\langle S_y\rangle
\\[1ex]
\langle S_z\rangle
\end{pmatrix}
=
\begin{pmatrix}
\langle x\rangle
\\[1ex]
\langle y\rangle
\\[1ex]
\langle z\rangle
\end{pmatrix}
=
\begin{pmatrix}
\frac{4}{9}
\\[1ex]
\frac{8}{9}
\\[1ex]
\frac{1}{9}
\end{pmatrix},\quad|\mathbf u|=1
\end{equation*}

To convert back to a spinor
\begin{equation*}
c_+=\sqrt{\frac{\langle z\rangle+1}{2}}=\frac{\sqrt5}{3}
\end{equation*}

and
\begin{equation*}
c_-=\sqrt{\frac{1-\langle z\rangle}{2}}
\frac{\langle x\rangle+i\langle y\rangle}{\sqrt{\langle x\rangle^2+\langle y\rangle^2}}
=\frac{2+4i}{3\sqrt5}
\end{equation*}

These values differ from the original $\chi$ but they do represent the same state.

\end{document}
