\input{preamble}

\FBOX{
8-1. A system is in the state $\psi=Y_{l,m}(\theta,\phi)$.
Calculate $\langle L_x\rangle$ and $\langle L_x^2\rangle$.
}

From equations (8.36) and (8.37) we have
\begin{align*}
(L_x+iL_y)Y_{l,m}&=c_{l,m}^+Y_{l,m+1}
\\
(L_x-iL_y)Y_{l,m}&=c_{l,m}^-Y_{l,m-1}
\end{align*}

Add together to obtain
\begin{equation*}
2L_xY_{l,m}=c_{l,m}^+Y_{l,m+1}+c_{l,m}^-Y_{l,m-1}
\tag{1}
\end{equation*}

By orthogonality of $Y_{l,m}$ we have
\begin{align*}
\langle L_x\rangle&=\oint Y_{lm}L_xY_{lm}\,d\Omega
\\
&=\frac{1}{2}\oint Y_{lm}\left(c_{l,m}^+Y_{l,m+1}+c_{l,m}^-Y_{l,m-1}\right)\,d\Omega
\\
&=0
\end{align*}

For $L_x^2$ we need to extend (1) as follows.
\begin{align*}
2L_xY_{l,m+1}&=c_{l,m+1}^+Y_{l,m+2}+c_{l,m+1}^-Y_{l,m}
\\
2L_xY_{l,m-1}&=c_{l,m-1}^+Y_{l,m}+c_{l,m-1}^-Y_{l,m-2}
\end{align*}

Then for $L_x^2$ we have
\begin{multline*}
L_x^2Y_{l,m}=L_xL_xY_{l,m}
=\frac{1}{2}L_x\left(c_{l,m}^+Y_{l,m+1}+c_{l,m}^-Y_{l,m-1}\right)
\\
{}=\frac{1}{4}\biggl(
 c_{l,m+1}^+c_{l,m}^+Y_{l,m+2}
+c_{l,m+1}^-c_{l,m}^+Y_{l,m}
\\
{}+c_{l,m-1}^+c_{l,m}^-Y_{l,m}
+c_{l,m-1}^-c_{l,m}^-Y_{l,m-2}
\biggr)
\end{multline*}

Hence
\begin{equation*}
\langle L_x^2\rangle=\oint Y_{l,m}L_x^2Y_{l,m}\,d\Omega=\frac{1}{4}
\left(c_{l,m+1}^-c_{l,m}^++c_{l,m-1}^+c_{l,m}^-\right)
\end{equation*}

Then by equations (8.40) and (8.41)
\begin{align*}
c_{l,m}^+c_{l,m+1}^-&=(l(l+1)-m(m+1))\hbar^2\tag{8.40}
\\
c_{l,m-1}^+c_{l,m}^-&=(l(l+1)-m(m-1))\hbar^2\tag{8.41}
\end{align*}

we have
\begin{equation*}
\langle L_x^2\rangle=\frac{1}{2}\left(l(l+1)-m^2\right)\hbar^2
\end{equation*}

Note that
\begin{multline*}
\langle L_x^2\rangle+\langle L_y^2\rangle+\langle L_z^2\rangle
=\frac{1}{2}\left(l(l+1)-m^2\right)\hbar^2
+\frac{1}{2}\left(l(l+1)-m^2\right)\hbar^2
+m^2\hbar^2
\\
{}=l(l+1)\hbar^2
\end{multline*}

as expected.

\end{document}
