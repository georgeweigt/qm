\input{preamble}

\fbox{\parbox{\dimexpr\linewidth-2\fboxsep-2\fboxrule}{
15.1.2
Given that
\begin{equation*}
\hat a^\dag\hat a|\psi_n\rangle=n|\psi_n\rangle
\end{equation*}
and
\begin{equation*}
\left[\hat a,\hat a^\dag\right]=\hat a\hat a^\dag-\hat a^\dag\hat a=1
\end{equation*}
show that
\begin{equation*}
\hat a^\dag\hat a\left(\hat a^\dag|\psi_n\rangle\right)
=(n+1)\left(\hat a^\dag|\psi_n\rangle\right)
\end{equation*}
}}

\bigskip
Noting that linear operators are associative we have
\begin{equation*}
\hat a^\dag\hat a\left(\hat a^\dag|\psi_n\rangle\right)
=\hat a^\dag(\hat a\hat a^\dag)|\psi_n\rangle
\end{equation*}

Then by the commutator relation given above we have
\begin{equation*}
\hat a^\dag\hat a\left(\hat a^\dag|\psi_n\rangle\right)
=\hat a^\dag(\hat a^\dag\hat a+1)|\psi_n\rangle
\end{equation*}

By the number operator given above we have
\begin{equation*}
\hat a^\dag\hat a\left(\hat a^\dag|\psi_n\rangle\right)
=\hat a^\dag(n+1)|\psi_n\rangle
\end{equation*}

Numbers commute with operators hence
\begin{equation*}
\hat a^\dag\hat a\left(\hat a^\dag|\psi_n\rangle\right)
=(n+1)\hat a^\dag|\psi_n\rangle
\end{equation*}

\end{document}
