\include{preamble}

\fbox{\parbox{\dimexpr\linewidth-2\fboxsep-2\fboxrule}{
8.3.1 Consider a ring of six identical ``atoms'' or ``unit cells'',
e.g., like a benzene ring, with a repeat length
(center to center distance between the atoms) of $0.5\,\text{nm}$.
Explicitly write out each of the different allowed Bloch forms
(i.e., $\psi(\nu)=u(\nu)\exp(ik\nu)$
for the effective one-dimensional electron wavefunctions $\psi(\nu)$,
where $\nu$ is the distance coordinate as we move round the ring,
and $u(\nu)$ is a function that is periodic with period $0.5\,\text{nm}$.
Be explicit about the numerical values and units of the allowed values of $k$.
}}

\bigskip
By hypothesis we have $N=6$ and $a=0.5\,\text{nm}$.

\bigskip
Then by equation (8.14) we have
\begin{equation*}
k=\frac{2\pi n}{Na}=\frac{2\pi n}{3}\,\text{nm}^{-1},\quad n=0,\pm1,\pm2\,\pm3
\end{equation*}

Hence the Bloch forms are
\begin{align*}
\psi(\nu)&=u(\nu) & &\text{for $n=0$}
\\
\psi(\nu)&=u(\nu)\exp\bigg(\pm\frac{2\pi i\nu}{3}\,\text{nm}^{-1}\bigg) & &\text{for $n=\pm1$}
\\
\psi(\nu)&=u(\nu)\exp\bigg(\pm\frac{4\pi i\nu}{3}\,\text{nm}^{-1}\bigg) & &\text{for $n=\pm2$}
\\
\psi(\nu)&=u(\nu)\exp\big(\pm2\pi i\nu\,\text{nm}^{-1}\big) & &\text{for $n=\pm3$}
\end{align*}

\end{document}
