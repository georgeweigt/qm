\input{preamble}

\FBOX{
6-9.
Suppose we introduce the fact that the atomic nucleus has a finite radius
given by
\begin{equation*}
r_N=1.2\,\text{fm}\times(\text{mass number})^{1/3}
\tag{6.56}
\end{equation*}

and assume that the nuclear charge is distributed approximately uniformly
in a sphere of this radius.
What is the effect of this assumption on the cross section for the
scattering of electrons by atoms at large
values of the momentum transfer $\breve p$?

\bigskip
Show how the nuclear radius can be determined along with some of the details
of the nuclear charge distribution by making use of this effect.
How large must the momentum $p$ of the incoming electrons be in order to
produce an appreciable effect? Would one observe more carefully the large
or small scattering angles? Why?
}

FIXME

\end{document}
