\input{preamble}

\FBOX{
6-4.
Using arguments similar to those leading to Eq.~(6.19),
show that the wave function $\psi(b)$ satisfies the integral equation
\begin{equation*}
\psi(b)=\phi(b)-\frac{i}{\hbar}
\int K_0(b,c)V(c)\psi(c)\,d\tau_c
\tag{6.26}
\end{equation*}

This integral equation is equivalent to the Schrodinger equation
\begin{equation*}
\frac{\partial\psi}{\partial t}=-\frac{i}{\hbar}
\left[-\frac{\hbar^2}{2m}\nabla^2\psi+V\psi\right]
\tag{6.27}
\end{equation*}

Working in one dimension only, show how the Schrodinger equation may
be deduced from the integral equation.
}

\begin{equation*}
K(b,a)=K_0(b,a)-\frac{i}{\hbar}\int K_0(b,c)V(c)K(c,a)\,d\tau_c
\end{equation*}

Recall
\begin{equation*}
\psi(b)=\int K(b,a)\psi(a)\,d\tau_a
\end{equation*}

Hence
\begin{equation*}
\psi(b)=\int K_0(b,a)\psi(a)\,d\tau_a
+\int\left(-\frac{i}{\hbar}\int K_0(b,c)V(c)K(c,a)\,d\tau_c\right)\psi(a)\,d\tau_a
\end{equation*}

Interchange the order of the integrals.
\begin{equation*}
\psi(b)=\int K_0(b,a)\psi(a)\,d\tau_a
-\frac{i}{\hbar}\int K_0(b,c)V(c)\left(\int K(c,a)\,d\tau_a\right)\,d\tau_c
\end{equation*}

Substitute $\psi(c)$ for $\int K(c,a)\,d\tau_a$.
\begin{equation*}
\psi(b)=\int K_0(b,a)\psi(a)\,d\tau_a
-\frac{i}{\hbar}\int K_0(b,c)V(c)\psi(c)\,d\tau_c
\end{equation*}

FIXME

\end{document}

