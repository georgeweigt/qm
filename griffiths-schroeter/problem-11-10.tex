\input{preamble}

% p. 417

\FBOX{
11.10
As a mechanism for downward transitions, spontaneous emission competes
with thermally stimulated emission (stimulated emission for which blackbody radiation is
the source). Show that at room temperature ($T=300\,\text{K}$) thermal stimulation dominates
for frequencies well below $5\times10^{12}\,\text{Hz}$, whereas spontaneous emission dominates for
frequencies well above $5\times10^{12}\,\text{Hz}$. Which mechanism dominates for visible light?
}

Let $A$ be the spontaneous emission rate and let $B\rho(\nu)$ be the stimulated emission rate.

\bigskip
We want to find frequency $\nu$ such that
\begin{equation*}
A=B\rho(\nu)
\end{equation*}

Substitute for $\rho(\nu)$.
\begin{equation*}
A=B\,\frac{A/B}{\exp\left(\frac{h\nu}{kT}\right)-1}
\end{equation*}

Hence
\begin{equation*}
\exp\left(\frac{h\nu}{kT}\right)=2
\end{equation*}

Take the log of both sides.
\begin{equation*}
\frac{h\nu}{kT}=\log2
\end{equation*}

Hence
\begin{equation*}
\nu=\frac{kT}{h}\log2
\end{equation*}

For $T=300\,\text{K}$ we have
\begin{equation*}
\nu=4.33\times10^{12}\,\text{Hz}
\end{equation*}

From equation (11.61)
\begin{equation*}
\frac{A}{B}\propto\nu^3
\end{equation*}

Hence $A$ dominates for large $\nu$.

\bigskip
Frequency of visible light is around $10^{15}\,\text{Hz}$.
This is well above $4.33\times10^{12}\,\text{Hz}$ hence spontaneous emission dominates.

\end{document}
