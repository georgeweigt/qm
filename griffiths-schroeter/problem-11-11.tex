\input{preamble}

\FBOX{
11.11
You could derive the spontaneous emission rate (Equation 11.63) without
the detour through Einstein's $A$ and $B$ coefficients if you knew the
ground state energy density of the electromagnetic field, $\rho_0(\omega)$,
for then it would simply be a case of stimulated
emission (Equation 11.54). To do this honestly would require quantum
electrodynamics, but if you are prepared to believe that the ground state
consists of {\it one photon in each classical mode}, then the derivation
is fairly simple:

\begin{itemize}

\item[(a)]
To obtain the classical modes, consider an empty cubical box, of side $l$,
with one corner at the origin.
Electromagnetic fields (in vacuum) satisfy the classical wave equation
\begin{equation*}
\left(\frac{1}{c^2}\frac{\partial^2}{\partial t^2}-\nabla^2\right)
f(x,y,z,t)=0,
\end{equation*}

where $f$ stands for any component of $\mathbf E$ or of $\mathbf B$.
Show that separation of variables,
and the imposition of the boundary condition $f=0$ on all six surfaces
yields the standing wave patterns
\begin{equation*}
f_{n_x,n_y,n_z}=A\cos(\omega t)
\sin\left(\frac{n_x\pi}{l}x\right)
\sin\left(\frac{n_y\pi}{l}y\right)
\sin\left(\frac{n_z\pi}{l}z\right),
\end{equation*}

with
\begin{equation*}
\omega=\frac{\pi c}{l}\sqrt{n_x^2+n_y^2+n_z^2}.
\end{equation*}

There are two modes for each triplet of positive integers
($n_x,n_y,n_z=1,2,3,\ldots$),
corresponding to the two polarization states.

\item[(b)]
The energy of a photon is $E=h\nu=\hbar\omega$ (Equation 4.92), so the energy
in the mode $(n_x,n_y,n_z)$ is
\begin{equation*}
E_{n_x,n_y,n_z}=2\frac{\pi\hbar c}{l}\sqrt{n_x^2+n_y^2+n_z^2}.
\end{equation*}

What, then, is the {\it total} energy per unit volume in the frequency
range $d\omega$, if each mode gets one photon?
Express your answer in the form
\begin{equation*}
\frac{1}{l^3}dE=\rho_0(\omega)\,d\omega
\end{equation*}

and read off $\rho_0(\omega)$.
{\it Hint:} refer to Figure 5.3.

\item[(c)]

\end{itemize}
}

(a) See Eigenmath script.

\end{document}
