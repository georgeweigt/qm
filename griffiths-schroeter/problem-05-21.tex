\input{preamble}

\FBOX{
5.21
The density of copper is $8.96\,\text{g/cm}^3$, and its atomic weight
is $63.5\,\text{g/mole}$.

\bigskip
(a) Calculate the Fermi energy for copper (Equation 5.54).
Assume $d=1$, and give your answer in electron volts.

\bigskip
(b) What is the corresponding electron velocity?
{\it Hint:} Set $E_F=(1/2)mv^2$. Is it safe
to assume the electrons in copper are nonrelativistic?

\bigskip
(c) At what temperature would the characteristic thermal energy
($k_BT$, where $k_B$ is the Boltzmann constant
and $T$ is the Kelvin temperature)
equal the Fermi energy, for copper?
{\it Comment:} This is called the {\bf Fermi temperature}, $T_F$.
As long as the {\it actual} temperature is substantially below the Fermi
temperature, the material can be regarded as ``cold,'' with most of the
electrons in the lowest accessible state.
Since the melting point of copper is $1356\,\text{K}$,
solid copper is {\it always} cold.

\bigskip
(d) Calculate the degeneracy pressure (Equation 5.57) of copper,
in the electron gas model.
}

(a) Avogadro's number.
\begin{equation*}
N_A=6.02\times10^{23}\,\text{atoms/mole}
\end{equation*}

Hence
\begin{equation*}
\frac{N}{V}=\frac{8.96\,\text{g/cm}^3\times(100\,\text{cm/meter})^3}{63.5\,\text{g/mole}}
\times6.02\times10^{23}\,\text{mole}^{-1}
=8.49\times10^{28}\,\text{meter}^{-3}
\end{equation*}

Each atom contributes $d=1$ electrons.
\begin{equation*}
\rho=\frac{Nd}{V}=8.49\times10^{28}\,\text{meter}^{-3}
\end{equation*}

Then for
\begin{align*}
\hbar&=1.05\times10^{-34}\,\text{joule second}
\\
m&=9.11\times10^{-31}\,\text{kilogram}
\end{align*}

we have
\begin{equation*}
E_F=\frac{\hbar^2}{2m}\left(3\pi^2\rho\right)^{2/3}=2.34\times10^{-19}\,\text{joule}
\end{equation*}

Convert to electron volts.
\begin{equation*}
E_F=\frac{2.34\times10^{-19}\,\text{joule}}{1.60\times10^{-19}\,\text{joule/electronvolt}}
=1.46\,\text{electronvolt}
\end{equation*}

(b) Not relativistic.
\begin{equation*}
v=\sqrt{2E_F/m}=717\,\text{kilometer/second},\quad
v\ll300{,}000\,\text{kilometer/second}
\end{equation*}

(c) Boltzmann constant.
\begin{equation*}
k_B=1.38\times10^{-23}\,\text{joule/kelvin}
\end{equation*}

Hence
\begin{equation*}
T_F=\frac{E_F}{k_B}=16{,}900\,\text{kelvin}
\end{equation*}

(d)
\begin{equation*}
P=\frac{\left(3\pi^2\right)^{2/3}\hbar^2}{5m}\rho^{5/3}=3.80\times10^{10}\,\text{newton/meter}^2
\end{equation*}

\end{document}
