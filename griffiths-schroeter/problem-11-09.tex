\input{preamble}

\FBOX{
11.9
The first term in Equation 11.32 comes from the $e^{i\omega t}/2$ part of
$\cos(\omega t)$, and the second term from $e^{-\omega t}/2$.
Thus dropping the first term is formally equivalent to writing
$\hat H'=(V/2)e^{-i\omega t}$, which is to say,
\begin{equation*}
H_{ba}'=\frac{V_{ba}}{2}e^{-i\omega t},\quad
H_{ab}'=\frac{V_{ab}}{2}e^{i\omega t}.
\tag{11.36}
\end{equation*}

(The latter is required to make the Hamiltonian matrix hermitian---or, if you prefer, to
pick out the dominant term in the formula analogous to Equation 11.32 for $c_a(t)$.) Rabi
noticed that if you make this so-called {\bf rotating wave approximation} at the {\it beginning} of
the calculation, Equation 11.17 can be solved exactly, with no need for perturbation theory,
and no assumption about the strength of the field.

\begin{itemize}

\item[(a)]
Solve Equation 11.17 in the rotating wave approximation (Equation 11.36), for the
usual initial conditions: $c_a(0)=1$, $c_b(0)=0$. Express your results
($c_a(t)$ and $c_b(t)$) in terms of the {\bf Rabi flopping frequency},
\begin{equation*}
\omega_r\equiv\frac{1}{2}\sqrt{(\omega-\omega_0)^2+(|V_{ab}|/\hbar)^2}.
\tag{11.37}
\end{equation*}

\item[(b)]
Determine the transition probability, $P_{a\rightarrow b}(t)$, and show that
it never exceeds 1.
Confirm that $|c_a(t)|^2+|c_b(t)|^2=1$.

\item[(c)]
Check that $P_{a\rightarrow b}(t)$ reduces to the perturbation theory result
(Equation 11.35) when the perturbation is ``small,'' and state precisely what
small {\it means} in this context, as a constraint on $V$.

\item[(d)]
At what time does the system first return to its initial state?
\end{itemize}
}

(a) Equation (11.17).
\begin{equation*}
\dot c_a=-\frac{i}{\hbar}H_{ab}'e^{-i\omega_0t}c_b,\quad
\dot c_b=-\frac{i}{\hbar}H_{ba}'e^{i\omega_0t}c_a
\tag{11.17}
\end{equation*}

Substitute for $H_{ab}'$ and $H_{ba}'$.
\begin{equation*}
\dot c_a=-\frac{i}{2\hbar}V_{ab}e^{-i(\omega_0-\omega)t}c_b,\quad
\dot c_b=-\frac{i}{2\hbar}V_{ba}e^{i(\omega_0-\omega)t}c_a
\end{equation*}

From problem 11.3
\begin{equation*}
c_b(t)=-\frac{2i}{\hbar}H_{ba}'
\frac{\sin\left(\tfrac{1}{2}kt\right)}{k}
\exp\left(\tfrac{i}{2}\omega_0t\right),\quad
k=\sqrt{\omega_0^2+\frac{4H_{ab}'H_{ba}'}{\hbar^2}}
\end{equation*}

Substitute $V_{ab}/2$ for $H_{ab}'$, $V_{ba}/2$ for $H_{ba}'$,
and substitute $\omega_0-\omega$ for $\omega_0$.
\begin{equation*}
c_b(t)=-\frac{i}{\hbar}V_{ba}
\frac{\sin\left(\tfrac{1}{2}kt\right)}{k}
\exp\left(\tfrac{i}{2}(\omega_0-\omega)t\right),\quad
k=\sqrt{(\omega_0-\omega)^2+\frac{|V_{ab}|^2}{\hbar^2}}
\end{equation*}

Substitute $2\omega_r$ for $k$.
\begin{equation*}
c_b(t)=-\frac{i}{\hbar}V_{ba}
\frac{\sin(\omega_rt)}{2\omega_r}
\exp\left(\tfrac{i}{2}(\omega_0-\omega)t\right)
\end{equation*}

Solve for $c_a(t)$.
\begin{equation*}
c_a(t)=\frac{2i\hbar}{V_{ba}}e^{-i(\omega_0-\omega)t}\dot c_b
=\left[\cos(\omega_rt)+\frac{i(\omega_0-\omega)}{2\omega_r}\sin(\omega_rt)\right]
\exp\left(-\tfrac{i}{2}(\omega_0-\omega)t\right)
\tag{1}
\end{equation*}

(b) The transition probability $P_{a\rightarrow b}(t)$ is equal to $|c_b(t)|^2$.
\begin{equation*}
P_{a\rightarrow b}(t)=|c_b(t)|^2=\frac{|V_{ab}|^2}{\hbar^2}
\frac{\sin^2(\omega_rt)}{4\omega_r^2}
\tag{2}
\end{equation*}

For $\omega=\omega_0$ we have
\begin{equation*}
\omega_r^2=\frac{|V_{ab}|^2}{4\hbar^2}
\end{equation*}

Hence for $\omega=\omega_0$
\begin{equation*}
P_{a\rightarrow b}(t)=\sin^2(\omega_rt)\le1
\end{equation*}

Total probability:
\begin{equation*}
|c_a(t)|^2+|c_b(t)|^2
=\cos^2(\omega_rt)+\frac{(\omega-\omega_0)^2+|V_{ab}|^2/\hbar^2}{4\omega_r^2}\sin^2(\omega_t)=1
\tag{3}
\end{equation*}

(c) Rewrite $\omega_r$ as
\begin{equation*}
\omega_r=\frac{1}{2\hbar}\sqrt{\hbar^2(\omega_0-\omega)^2+|V_{ab}|^2}
\end{equation*}

and note that for
\begin{equation*}
\hbar^2(\omega_0-\omega)^2\gg|V_{ab}|^2
\end{equation*}

we have
\begin{equation*}
\omega_r\approx\tfrac{1}{2}|\omega_0-\omega|
\end{equation*}

Substitute this approximation into the formula for $c_b(t)$ to obtain
\begin{equation*}
c_b(t)=-\frac{i}{\hbar}V_{ba}
\frac{\sin\left(\tfrac{1}{2}|\omega_0-\omega|t\right)}{|\omega_0-\omega|}
\exp\left(\tfrac{i}{2}(\omega_0-\omega)t\right)
\end{equation*}

This is equivalent to $c_b(t)$ obtained from first order perturbation expansion.

\end{document}
