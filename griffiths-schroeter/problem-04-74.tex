\input{preamble}

\FBOX{
4.74
Neither Example 4.4 nor Problem 4.73 actually solved the Schr\"odinger
equation for the Stern-Gerlach experiment. In this problem we will see how to set
up that calculation. The Hamiltonian for a neutral, spin-1/2 particle traveling through
a Stern-Gerlach device is
\begin{equation*}
H=\frac{p^2}{2m}-\gamma\mathbf B\cdot\mathbf S
\end{equation*}

where $\mathbf B$ is given by Equation 4.169.
The most general wave function for a spin-1/2
particle---including both spatial and spin degrees of freedom---is
\begin{equation*}
\mathbf\Psi(\mathbf r,t)=\Psi_+(\mathbf r,t)\chi_++\Psi_-(\mathbf r,t)\chi_-
\end{equation*}

(a) Put $\mathbf\Psi(\mathbf r,t)$ into the Schr\"odinger equation
\begin{equation*}
H\mathbf\Psi=i\hbar\frac{\partial}{\partial t}\mathbf\Psi
\end{equation*}

to obtain a pair of coupled equations for $\Psi_\pm$.
}

From equations (4.140) and (4.141) we have
\begin{equation*}
\chi_+=\begin{pmatrix}1\\0\end{pmatrix},\quad
\chi_-=\begin{pmatrix}0\\1\end{pmatrix}
\end{equation*}

Hence
\begin{equation*}
\mathbf\Psi=\begin{pmatrix}\Psi_+\\\Psi_-\end{pmatrix}
\end{equation*}

From equation (4.169) we have for a beam in the $y$ direction
\begin{equation*}
\mathbf B(x,y,z)=-\alpha x\begin{pmatrix}1\\0\\0\end{pmatrix}
+(B_0+\alpha z)\begin{pmatrix}0\\0\\1\end{pmatrix}
\end{equation*}

From equation (4.148)
\begin{equation*}
\mathbf S=\frac{\hbar}{2}\begin{pmatrix}\sigma_x\\\sigma_y\\\sigma_z\end{pmatrix}
\end{equation*}

Hence
\begin{equation*}
\mathbf B\cdot\mathbf S
=-\frac{\hbar}{2}\alpha x\sigma_x
+\frac{\hbar}{2}(B_0+\alpha z)\sigma_z
\end{equation*}

Noting that
\begin{equation*}
\sigma_x\mathbf\Psi
=\begin{pmatrix}0&1\\1&0\end{pmatrix}\mathbf\Psi
=\begin{pmatrix}\Psi_-\\\Psi_+\end{pmatrix}
\end{equation*}

and
\begin{equation*}
\sigma_z\mathbf\Psi
=\begin{pmatrix}1&0\\0&-1\end{pmatrix}\mathbf\Psi
=\begin{pmatrix}\Psi_+\\-\Psi_-\end{pmatrix}
\end{equation*}

we have
\begin{equation*}
\mathbf B\cdot\mathbf S
=-\frac{\hbar}{2}\alpha x\begin{pmatrix}\Psi_-\\\Psi_+\end{pmatrix}
+\frac{\hbar}{2}(B_0+\alpha z)\begin{pmatrix}\Psi_+\\-\Psi_-\end{pmatrix}
\end{equation*}

For the kinetic energy term in $H$
\begin{align*}
\frac{p^2}{2m}\mathbf\Psi
&=-\frac{\hbar^2}{2m}\nabla^2\mathbf\Psi
\\
&=-\frac{\hbar^2}{2m}\nabla^2\Psi_+\chi_+ % \begin{pmatrix}1\\0\end{pmatrix}
-\frac{\hbar^2}{2m}\nabla^2\Psi_-\chi_- % \begin{pmatrix}0\\1\end{pmatrix}
\\
&=-\frac{\hbar^2}{2m}\begin{pmatrix}\nabla^2\Psi_+\\\nabla^2\Psi_-\end{pmatrix}
\end{align*}

From the Schr\"odinger equation
\begin{equation*}
\left(\frac{p^2}{2m}-\gamma\mathbf B\cdot\mathbf S\right)\mathbf\Psi
=i\hbar\frac{\partial}{\partial t}\mathbf\Psi
\end{equation*}

we obtain
\begin{equation*}
-\frac{\hbar^2}{2m}\begin{pmatrix}\nabla^2\Psi_+\\\nabla^2\Psi_-\end{pmatrix}
+\frac{\hbar}{2}\gamma\alpha x\begin{pmatrix}\Psi_-\\\Psi_+\end{pmatrix}
-\frac{\hbar}{2}\gamma(B_0+\alpha z)\begin{pmatrix}\Psi_+\\-\Psi_-\end{pmatrix}
=i\hbar\frac{\partial}{\partial t}\begin{pmatrix}\Psi_+\\\Psi_-\end{pmatrix}
\end{equation*}

Then in component form we obtain the coupled equations
\begin{align*}
-\frac{\hbar^2}{2m}\nabla^2\Psi_+
+\frac{\hbar}{2}\gamma\alpha x\Psi_-
-\frac{\hbar}{2}\gamma(B_0+\alpha z)\Psi_+
&=i\hbar\frac{\partial}{\partial t}\Psi_+
\\[1ex]
-\frac{\hbar^2}{2m}\nabla^2\Psi_-
+\frac{\hbar}{2}\gamma\alpha x\Psi_+
+\frac{\hbar}{2}\gamma(B_0+\alpha z)\Psi_-
&=i\hbar\frac{\partial}{\partial t}\Psi_-
\end{align*}


\end{document}
