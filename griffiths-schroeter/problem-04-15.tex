\input{preamble}

\FBOX{
4.15
(a) Find $\langle r\rangle$ and $\langle r^2\rangle$ for an electron in the
ground state of hydrogen.
Express your answers in terms of the Bohr radius.

\bigskip
(b) Find $\langle x\rangle$ and $\langle x^2\rangle$ for an electron in the
ground state of hydrogen.
{\it Hint:} This requires no new integration---note that
$r^2=x^2+y^2+z^2$, and exploit the symmetry of the ground state.

\bigskip
(c) Find $\langle x^2\rangle$ in the state $n=2$, $\ell=1$, $m=1$.
{\it Hint:} this state is {\it not} symmetrical in $x$, $y$, $z$.
Use $x=r\sin\theta\cos\phi$.
}

(a)
\begin{equation*}
\psi_{100}(r,\theta,\phi)=\frac{1}{\sqrt{\pi a_0^3}}\exp\left(-\frac{r}{a_0}\right)
\end{equation*}

For $\langle r\rangle$ we have
\begin{align*}
\langle r\rangle&=\int_0^{2\pi}\int_0^\pi\int_0^\infty r|\psi_{100}|^2r^2\sin\theta\,dr\,d\theta\,d\phi
\\
&=\frac{1}{\pi a_0^3}\int_0^{2\pi}\int_0^\pi\int_0^\infty
r^3\exp\left(-\frac{2r}{a_0}\right)\sin\theta\,dr\,d\theta\,d\phi
\end{align*}

Integrate over $\phi$ (multiply by $2\pi$).
\begin{equation*}
\langle r\rangle=\frac{2}{a_0^3}\int_0^\pi\int_0^\infty
r^3\exp\left(-\frac{2r}{a_0}\right)\sin\theta\,dr\,d\theta
\end{equation*}

Transform the integral over $\theta$ to an integral over $y$ where
$y=\cos\theta$ and $dy=-\sin\theta\,d\theta$.
The minus sign in $dy$ is canceled by interchanging integration limits
$\cos0=1$ and $\cos\pi=-1$.
\begin{equation*}
\langle r\rangle=\frac{2}{a_0^3}\int_{-1}^1\int_0^\infty
r^3\exp\left(-\frac{2r}{a_0}\right)\,dr\,dy
\end{equation*}

Integrate over $y$ (multiply by 2).
\begin{equation*}
\langle r\rangle=\frac{4}{a_0^3}\int_0^\infty
r^3\exp\left(-\frac{2r}{a_0}\right)\,dr
\end{equation*}

Solve the integral over $r$.
\begin{equation*}
\langle r\rangle=\tfrac{3}{2}a_0\tag{1}
\end{equation*}

For $\langle r^2\rangle$ we have
\begin{equation*}
\langle r^2\rangle=\frac{4}{a_0^3}\int_0^\infty
r^4\exp\left(-\frac{2r}{a_0}\right)\,dr=3a_0^2\tag{2}
\end{equation*}

(b) By symmetry we have $\langle x\rangle=0$ and
\begin{equation*}
\langle r^2\rangle=\langle x^2\rangle+\langle y^2\rangle+\langle z^2\rangle=3\langle x^2\rangle
\end{equation*}

Hence
\begin{equation*}
\langle x^2\rangle=\tfrac{1}{3}\langle r^2\rangle=a_0^2
\end{equation*}

(c)
\begin{equation*}
\psi_{211}=-\frac{r\sin\theta}{8\sqrt{\pi a_0^5}}\exp\left(-\frac{r}{2a_0}+i\phi\right)
\end{equation*}

For $\langle x^2\rangle$ we have
\begin{align*}
\langle x^2\rangle&=\int_0^{2\pi}\int_0^\pi\int_0^\infty
(r\sin\theta\cos\phi)^2
|\psi_{211}|^2r^2\sin\theta\,dr\,d\theta\,d\phi
\\
&=\frac{1}{64\pi a_0^5}\int_0^{2\pi}\int_0^\pi\int_0^\infty
r^6\exp\left(-\frac{r}{a_0}\right)\sin^5\theta\cos^2\phi\,dr\,d\theta\,d\phi
\end{align*}

Integrate over $\phi$ (multiply by $\pi$).
\begin{equation*}
\langle x^2\rangle=\frac{1}{64a_0^5}\int_0^\pi\int_0^\infty
r^6\exp\left(-\frac{r}{a_0}\right)\sin^3\theta\,dr\,d\theta
\end{equation*}

Integrate over $\theta$ (multiply by $\frac{16}{15}$).
\begin{equation*}
\langle x^2\rangle=\frac{1}{60a_0^5}\int_0^\infty
r^6\exp\left(-\frac{r}{a_0}\right)\,dr
\end{equation*}

Integrate over $r$.
\begin{equation*}
\langle x^2\rangle=12a_0^2
\end{equation*}

\end{document}
