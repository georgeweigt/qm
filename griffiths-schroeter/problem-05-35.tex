\input{preamble}

\FBOX{
5.35
Certain cold stars (called {\bf white dwarfs}) are stabilized against gravitational
collapse by the degeneracy pressure of their electrons (Equation 5.57). Assuming
constant density, the radius $R$ of such an object can be calculated as follows:

\bigskip
(a) Write the total electron energy (Equation 5.56) in terms of the radius, the number
of nucleons (protons and neutrons) $N$, the number of electrons per nucleon $d$, and
the mass of the electron $m$. {\it Beware:} In this problem we are recycling the letters $N$
and $d$ for a slightly different purpose than in the text.

\bigskip
(b) Look up, or calculate, the gravitational energy of a uniformly dense sphere.
Express your answer in terms of $G$ (the constant of universal gravitation), $R$, $N$,
and $M$ (the mass of a nucleon). Note that the gravitational energy is {\it negative}.

\bigskip
(c) Find the radius for which the total energy, (a) plus (b), is a minimum. {\it Answer:}
\begin{equation*}
R=\left(\frac{9\pi}{4}\right)^{2/3}\frac{\hbar^2d^{5/3}}{GmM^2N^{1/3}}
\end{equation*}

(Note that the radius {\it decreases} as the total mass {\it increases!}) Put in the actual
numbers, for everything except $N$, using $d=1/2$ (actually, $d$ decreases a bit as
the atomic number increases, but this is close enough for our purposes). {\it Answer:}
$R =7.6\times10^{25} N^{-1/3}\,\text{m}$.

\bigskip
(d) Determine the radius, in kilometers, of a white dwarf with the mass of the sun.

\bigskip
(e) Determine the Fermi energy, in electron volts, for the white dwarf in (d), and
compare it with the rest energy of an electron. Note that this system is getting
dangerously relativistic (see Problem 5.36).
}

(a) In equation (5.56) $N$ is the number of atoms and each atom contributes $d$ electrons.
In this problem $N$ is the number of nucleons and $d$ is the number of electrons per nucleon.
\begin{equation*}
E_{\rm tot}=\frac{\hbar^2\left(3\pi^2Nd\right)^{5/3}}{10\pi^2m}V^{-2/3}\tag{5.56}
\end{equation*}

By $V=4\pi R^3/3$ we have
\begin{equation*}
E_{\rm tot}=\frac{\hbar^2\left(3\pi^2Nd\right)^{5/3}}{10\pi^2m}
\left(\frac{4\pi R^3}{3}\right)^{-2/3}
=\frac{9}{20}\left(\frac{3\pi^2}{2}\right)^{1/3}\frac{\hbar^2(Nd)^{5/3}}{mR^2}
\end{equation*}

(b)
\begin{equation*}
U=-\frac{3G(NM)^2}{5R}
\end{equation*}

(c) We have
\begin{align*}
\frac{dE_{\rm tot}}{dR}&=-\frac{9}{10}\left(\frac{3\pi^2}{2}\right)^{1/3}
\frac{\hbar^2(Nd)^{5/3}}{mR^3}
\\
\frac{dU}{dR}&=\frac{3G(MN)^2}{5R^2}
\end{align*}

Find $R$ such that
\begin{equation*}
\frac{dE_{\rm tot}}{dR}+\frac{dU}{dR}=0
\end{equation*}

Substitute and multiply both sides by $R^3$.
\begin{equation*}
-\frac{9}{10}\left(\frac{3\pi^2}{2}\right)^{1/3}
\frac{\hbar^2(Nd)^{5/3}}{m}+\frac{3G(MN)^2}{5}R=0
\end{equation*}

Hence
\begin{equation*}
R=\frac{9}{10}\left(\frac{3\pi^2}{2}\right)^{1/3}\frac{\hbar^2(Nd)^{5/3}}{m}\frac{5}{3G(MN)^2}
=\left(\frac{9\pi}{4}\right)^{2/3}\frac{\hbar^2d^{5/3}}{GmM^2N^{1/3}}
\tag{1}
\end{equation*}

(d) For $M_{\odot}=1.98892\times10^{30}\,\text{kg}$
and $M=m_p=1.67\times10^{-27}\,\text{kg}$ we have
\begin{equation*}
N=\frac{M_{\odot}}{M}=1.19\times10^{57}
\end{equation*}

Hence
\begin{equation*}
R=7160\,\text{km}
\end{equation*}

(e) The fermi energy is
\begin{equation*}
E_F=\frac{\hbar^2}{2m}\left(\frac{3\pi^2Nd}{V}\right)^{2/3}=194\,\text{keV}
\end{equation*}

The rest energy of an electron is
\begin{equation*}
m_ec^2=511\,\text{keV}
\end{equation*}

\end{document}
