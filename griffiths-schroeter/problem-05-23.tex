\input{preamble}

\FBOX{
5.23
The {\bf bulk modulus} of a substance is the ratio of a small decrease in pressure
to the resulting fractional increase in volume:
\begin{equation*}
B=-V\frac{dP}{dV}.
\end{equation*}

Show that $B=(5/3)P$, in the free electron gas model, and use your result in Problem
5.21(d) to estimate the bulk modulus of copper. {\it Comment:} the observed value is
$13.4\times10^{10}\,\text{N/m}^2$, but don't expect perfect agreement---after all, we're
neglecting all electron-nucleus and electron-electron forces! Actually, it is rather
surprising that this calculation comes as close as it {\it does}.
}

\begin{equation*}
P=\frac{\left(3\pi^2\right)^{2/3}\hbar^2}{5m}
\left(\frac{Nd}{V}\right)^{5/3}
\end{equation*}

Noting that
\begin{equation*}
\frac{d}{dV}V^{-5/3}=-\frac{5}{3}V^{-8/3}
\end{equation*}

we have
\begin{equation*}
\frac{dP}{dV}=-\frac{5}{3}PV^{-1}
\tag{1}
\end{equation*}

Hence
\begin{equation*}
B=-V\frac{dP}{dV}=\frac{5}{3}P
\end{equation*}

From problem 5.21(d)
\begin{equation*}
P=3.80\times10^{10}\,\text{N/m}^2
\end{equation*}

Hence
\begin{equation*}
B=\frac{5}{3}P=6.33\times10^{10}\,\text{N/m}^2
\end{equation*}

\end{document}
