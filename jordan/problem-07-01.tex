\input{preamble}

\fbox{\parbox{\dimexpr\linewidth-2\fboxsep-2\fboxrule}{
7-1.
Two quantities are represented by the matrices
\begin{equation*}
M=\begin{pmatrix}3&0&-i\\0&1&0\\i&0&3\end{pmatrix},\quad
N=\begin{pmatrix}3&0&2i\\0&7&0\\-2i&0&3\end{pmatrix}
\end{equation*}

The possible values of the quantity represented by $M$
are 1, 2, and 4.
What are the possible values of the quantity represented by $N$?
Explain how you know that.
}}

\bigskip
There is no relation between the values represented by $M$
and the values represented by $N$ because the matrices commute.
\begin{equation*}
MN=NM
\end{equation*}

\end{document}
