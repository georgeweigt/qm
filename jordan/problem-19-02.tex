\input{preamble}

\fbox{\parbox{\dimexpr\linewidth-2\fboxsep-2\fboxrule}{
19-2.
Use the results of the last problem to find formulas for the period of
revolution (that is the time one revolution takes) and the frequency
of revolution (that is the number of revolutions per unit time) for the
circular Bohr orbit for each $n$. Again, write the answers in terms of
$n$, $\hbar$, $n$, and $Ze^2$ only.
}}

\bigskip
The time for one revolution is
\begin{equation*}
t=\frac{2\pi r}{v}
\end{equation*}

From the last problem
\begin{equation*}
r=\frac{\hbar^2n^2}{mZe^2},\quad v=\frac{Ze^2}{n\hbar}
\end{equation*}

Hence
\begin{equation*}
t=\frac{2\pi n^3\hbar^3}{m(Ze^2)^2}\tag{1}
\end{equation*}

and
\begin{equation*}
\nu=t^{-1}=\frac{m(Ze^2)^2}{2\pi\hbar^3n^3}
\end{equation*}

\end{document}
