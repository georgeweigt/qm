\input{preamble}

\fbox{\parbox{\dimexpr\linewidth-2\fboxsep-2\fboxrule}{
20-6.
Suppose the quantity represented by
${L_1}^2+{L_2}^2+{L_3}^2$ has the value $2\hbar^2$, that $\langle L_3\rangle$ is 0,
and that $\langle{L_3}^2\rangle$ is $\hbar^2/2$.
What are the possible values for the quantity represented by $L_3$?
What are the probabilities for these values?
}}

\bigskip
From $j(j+1)\hbar^2=2\hbar^2$ we have $j=1$.
The possible values for $L_3$ are $k\hbar$ where
\begin{equation*}
k=-1,0,1
\end{equation*}

From the expectation value
\begin{equation*}
\langle L_3\rangle=\rho(-1)(-\hbar)+\rho(1)\hbar=0
\end{equation*}

we have $\rho(-1)=\rho(1)$ and from
\begin{equation*}
\langle{L_3}^2\rangle=\rho(-1)\hbar^2+\rho(1)\hbar^2=\tfrac{1}{2}\hbar^2
\end{equation*}

we have
\begin{equation*}
\rho(-1)=\rho(1)=\tfrac{1}{4}
\end{equation*}

By total probability
\begin{equation*}
\rho(0)=1-\rho(-1)-\rho(1)=\tfrac{1}{2}
\end{equation*}

\end{document}
