\input{preamble}

\fbox{\parbox{\dimexpr\linewidth-2\fboxsep-2\fboxrule}{
10-2.
For each of the matrices (a)--(f):

(i) determine whether the matrix represents a physical quantity; if it does, then

(ii) find the possible values of the quantity;

(iii) determine whether the quantity is real; and

(iv) determine whether the matrix has an inverse.
\begin{align*}
(a)&\quad 2+2\Sigma_1 & (d)&\quad i+3\Sigma_1+4\Sigma_2
\\
(b)&\quad 3\Sigma_1+4\Sigma_2 & (e)&\quad i(3\Sigma_1+4\Sigma_2)
\\
(c)&\quad 5+3\Sigma_1+4\Sigma_2 & (f)&\quad 5\Sigma_1+i4\Sigma_2
\end{align*}
}}

\bigskip
(a) The matrix $2+2\Sigma_1$ represents a physical quantity with
\begin{equation*}
x_0=2,\quad r=\sqrt{2^2+0^2+0^2}=2
\end{equation*}

The possible values are
\begin{equation*}
\begin{aligned}
x_0+r&=4
\\
x_0-r&=0
\end{aligned}
\end{equation*}

This quantity is real.
This matrix does not have an inverse because one of the values is zero.

\bigskip
(b) The matrix $3\Sigma_1+4\Sigma_2$ represents a physical quantity with
\begin{equation*}
x_0=0,\quad r=\sqrt{3^2+4^2+0^2}=5
\end{equation*}

The possible values are
\begin{equation*}
\begin{aligned}
x_0+r&=5
\\
x_0-r&=-5
\end{aligned}
\end{equation*}

This quantity is real.
This matrix has an inverse because both values are nonzero.

\bigskip
(c) The matrix $5+3\Sigma_1+4\Sigma_2$ represents a physical quantity with
\begin{equation*}
x_0=5,\quad r=\sqrt{3^2+4^2+0^2}=5
\end{equation*}

The possible values are
\begin{equation*}
\begin{aligned}
x_0+r&=10
\\
x_0-r&=0
\end{aligned}
\end{equation*}

This quantity is real.
This matrix does not have an inverse because one of the values is zero.

\bigskip
(d) The matrix $i+3\Sigma_1+4\Sigma_2$ represents a physical quantity with
\begin{equation*}
z_0=i,\quad c=1,\quad r=\sqrt{3^2+4^2+0^2}=5
\end{equation*}

The possible values are
\begin{equation*}
\begin{aligned}
z_0+cr&=i+5
\\
z_0-cr&=i-5
\end{aligned}
\end{equation*}

This quantity is not real.
This matrix has an inverse because both values are nonzero.

\bigskip
(e) The matrix $i(3\Sigma_1+4\Sigma_2)$ represents a physical quantity with
\begin{equation*}
z_0=0,\quad c=i,\quad r=\sqrt{3^2+4^2+0^2}=5
\end{equation*}

The possible values are
\begin{equation*}
\begin{aligned}
z_0+cr&=5i
\\
z_0-cr&=-5i
\end{aligned}
\end{equation*}

This quantity is not real.
This matrix has an inverse because both values are nonzero.

\bigskip
(f) The matrix $5\Sigma_1+i4\Sigma_2$ does not represent a physical quantity
because the imaginary unit cannot be factored as $c$.

\end{document}
