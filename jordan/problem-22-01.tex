\input{preamble}

\fbox{\parbox{\dimexpr\linewidth-2\fboxsep-2\fboxrule}{
22-1.
It can be shown, as Pauli did, that
\begin{multline*}
\tfrac{1}{2}(A_1Q_1+Q_1A_1+A_2Q_2+Q_2A_2+A_3Q_3+Q_3A_3)
\\
=-\frac{1}{mZe^2}\left({L_1}^2+{L_2}^2+{L_3}^2+\frac{3}{2}\hbar^2\right)
+R
\end{multline*}

Use this and the result of Problem 17-6 to find
$\langle R\rangle$ for a state where each of the quantities
represented by $L_1$, $L_2$, $L_3$ and $A_1$, $A_2$, $A_3$ has the
value 0.
Show this occurs when the $\varepsilon_n$ for $n=1$;
the results of Problems 17-4 and 20-7 can be used here.
Compare $\langle R\rangle$ found here with the
radius of the Bohr orbit found in Problem 19-1.
}}

\bigskip
For $A_1$, $A_2$, and $A_3$ with definite value zero we have zero variance hence
\begin{equation*}
\langle{A_1}^2\rangle=\langle A_1\rangle^2=0,\quad
\langle{A_2}^2\rangle=\langle A_2\rangle^2=0,\quad
\langle{A_3}^2\rangle=\langle A_3\rangle^2=0
\end{equation*}

Then by problem 17-6 we have
\begin{equation*}
\langle A_1Q_1+Q_1A_1\rangle=0,\quad
\langle A_2Q_2+Q_2A_2\rangle=0,\quad
\langle A_3Q_3+Q_3A_3\rangle=0
\end{equation*}

For $L_1$, $L_2$, and $L_3$ with definite value zero we have zero variance hence
\begin{equation*}
\langle{L_1}^2\rangle=\langle L_1\rangle^2=0,\quad
\langle{L_2}^2\rangle=\langle L_2\rangle^2=0,\quad
\langle{L_3}^2\rangle=\langle L_3\rangle^2=0
\end{equation*}

Hence
\begin{equation*}
\langle R\rangle=\frac{3}{2}\frac{\hbar^2}{mZe^2}
\end{equation*}

For $\langle L^2\rangle=0$ we have $l=0$ and $j=0$ hence
\begin{equation*}
n=2j+1=1
\end{equation*}

From problem 19-1 for $n=1$ we have
\begin{equation*}
r=\frac{\hbar^2}{mZe^2}
\end{equation*}

Note that $\langle R\rangle$ is the expected radius and $r$ is the most likely radius.

\bigskip
The result for $Z=1$ should be
\begin{equation*}
\langle R\rangle=\tfrac{3}{2}a_0
\end{equation*}

where
\begin{equation*}
a_0=\frac{4\pi\epsilon_0\hbar^2}{me^2}
\end{equation*}

The author is using $4\pi\epsilon_0=1$ hence
\begin{equation*}
\langle R\rangle=\frac{3}{2}\frac{\hbar^2}{mZe^2}
\end{equation*}

\end{document}
