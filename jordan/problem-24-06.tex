\input{preamble}

\fbox{\parbox{\dimexpr\linewidth-2\fboxsep-2\fboxrule}{
24-6.
To describe the spin and magnetic moment
of a particle such as an electron or proton
together with the motion in three-dimensional
space, we need matrices $\Sigma_1$, $\Sigma_2$, $\Sigma_3$ as
well as matrices $Q_1$, $Q_2$, $Q_3$ and $P_1$, $P_2$, $P_3$ that
represent position and momentum. The
matrices $\Sigma_1$, $\Sigma_2$, $\Sigma_3$ commute with $Q_1$, $Q_2$, $Q_3$
and $P_1$, $P_2$, $P_3$, and the multiplication rules for
$Q_1$, $Q_2$, $Q_3$ and $P_1$, $P_2$, $P_3$ are the same as before.
The total angular momentum is the sum of
the orbital angular momentum and spin
angular momentum. Then
\begin{align*}
J_1&=\frac{1}{\hbar}L_1+\tfrac{1}{2}\Sigma_1,
\\
J_2&=\frac{1}{\hbar}L_2+\tfrac{1}{2}\Sigma_2,
\\
J_3&=\frac{1}{\hbar}L_3+\tfrac{1}{2}\Sigma_3.
\end{align*}

Show that these matrices satisfy the commutation
relations characteristic of matrices $J_1$, $J_2$,
$J_3$ that correspond to rotations. The spin
angular momentum represented by the matrices
\begin{equation*}
\tfrac{1}{2}\hbar\Sigma_1,\quad
\tfrac{1}{2}\hbar\Sigma_2,\quad
\tfrac{1}{2}\hbar\Sigma_3;
\end{equation*}

the magnetic moment represented by
\begin{equation*}
\mu\Sigma_1,\quad
\mu\Sigma_2,\quad
\mu\Sigma_3;
\end{equation*}

the position represented by $Q_1$, $Q_2$, $Q_3$; the
momentum represented by $P_1$, $P_2$, $P_3$; and the
orbital angular momentum represented by $L_1$,
$L_2$, $L_3$ are vector quantities. Show that with
each of these the matrices $J_1$, $J_2$, $J_3$ satisfy the
commutation relations that characterize rotations
of a vector quantity.
}}

\bigskip
We have
\begin{multline*}
J_1J_2-J_2J_1=\frac{1}{\hbar^2}L_1L_2+\frac{1}{2\hbar}L_1\Sigma_2+\frac{1}{2\hbar}\Sigma_1L_2
+\tfrac{1}{4}\Sigma_1\Sigma_2
\\
-\frac{1}{\hbar^2}L_2L_1-\frac{1}{2\hbar}L_2\Sigma_1-\frac{1}{2\hbar}\Sigma_2L_1
-\tfrac{1}{4}\Sigma_2\Sigma_1
\\
=\frac{1}{\hbar^2}(L_1L_2-L_2L_1)+\tfrac{1}{4}(\Sigma_1\Sigma_2-\Sigma_2\Sigma_1)
=\frac{i}{\hbar}L_3+\frac{i}{2}\Sigma_3=iJ_3
\end{multline*}

See the Eigenmath solution for the remaining commutation relations.

\end{document}
