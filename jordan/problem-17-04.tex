\input{preamble}

\fbox{\parbox{\dimexpr\linewidth-2\fboxsep-2\fboxrule}{
17-4.
Suppose $A$ represents a real quantity.
Then $\langle(A-\langle A\rangle)^2\rangle$ is real and non-negative.
Show this implies
\begin{equation*}
\langle A^2\rangle\ge\langle A\rangle^2.
\end{equation*}

Show that if $\langle A^2\rangle$ is 0, then $\langle A\rangle$ is 0 and
$\langle(A-\langle A\rangle)^2\rangle$ is 0, which means the quantity
represented by $A$ has the definite value 0.
}}

\bigskip
Recall that
\begin{equation*}
\langle(A-\langle A\rangle)^2\rangle=\langle A^2\rangle-\langle A\rangle^2
\end{equation*}

Hence if $\langle(A-\langle A\rangle)^2\rangle$ is non-negative then
$\langle A^2\rangle-\langle A\rangle^2$ is also non-negative.
\begin{equation*}
\langle A^2\rangle-\langle A\rangle^2\ge0
\end{equation*}

Hence
\begin{equation*}
\langle A^2\rangle\ge\langle A\rangle^2\tag{1}
\end{equation*}

Since $A$ represents a real quantity then $\langle A\rangle^2$ is a non-negative quantity.
\begin{equation*}
\langle A\rangle^2\ge0\tag{2}
\end{equation*}

If $\langle A^2\rangle=0$ then by (1) and (2)
\begin{equation*}
0\ge\langle A\rangle^2\ge0
\end{equation*}

Hence $\langle A\rangle^2=0$ and $\langle(A-\langle A\rangle)^2\rangle=0$.

\end{document}
