\input{preamble}

\fbox{\parbox{\dimexpr\linewidth-2\fboxsep-2\fboxrule}{
25-1.
Let
\begin{equation*}
U=1-i\varepsilon K_2-\tfrac{1}{2}\varepsilon^2{K_2}^2.
\end{equation*}

Calculate $U^{-1}L_3U$. Do this two different ways.
First use
\begin{equation*}
L_3=Q_1P_2-Q_2P_1,
\end{equation*}

the fact that $K_2$ and $U$ commute with $P_1$ and
$P_2$, and the formulas for $U^{-1}Q_1U$ and $U^{-1}Q_2U$.
Then use
\begin{equation*}
U^{-1}L_3U=L_3-i\varepsilon(L_3K_2-K_2L_3)
-\tfrac{1}{2}\varepsilon^2[(L_3K_2-K_2L_3)K_2-K_2(L_3K_2-K_2L_3)]
\end{equation*}

and the commutation relation
\begin{equation*}
L_3K_2-K_2L_3=-iP_1.
\end{equation*}

If you get the same answer, it shows the commutation
relation does correspond to the way
angular momentum is changed by a translation.
}}

\bigskip
For the first method we have
\begin{align*}
U^{-1}L_3U&=U^{-1}Q_1P_2U-U^{-1}Q_2P_1U
\\
&=U^{-1}Q_1UP_2-U^{-1}Q_2UP_1
\\
&=Q_1P_2-(Q_2+\varepsilon)P_1
\\
&=L_3-\varepsilon P_1
\end{align*}

For the second method we have
\begin{align*}
U^{-1}L_3U
&=L_3-i\varepsilon(-iP_1)-\tfrac{1}{2}\varepsilon^2[(-iP_1)K_2-K_2(-iP_1)]
\\
&=L_3-i\varepsilon P_1
\end{align*}

\end{document}
