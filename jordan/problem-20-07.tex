\input{preamble}

\fbox{\parbox{\dimexpr\linewidth-2\fboxsep-2\fboxrule}{
20-7.
Use the result of Problem 17-4 to show that if
$\langle{L_1}^2+{L_2}^2+{L_3}^2\rangle$ is 0 then each of the
quantities represented by $L_1$, $L_2$, $L_3$ has the value 0.
This happens when the value $j(j+1)\hbar^2$ of the
quantity represented by ${L_1}^2+{L_2}^2+{L_3}^2$ is 0.
}}

\bigskip
From problem 17-4 if $A$ represents a real quantity then $\langle A^2\rangle\ge0$ and
if $\langle A^2\rangle=0$ then the quantity represented by $A$ has the definite value of zero.

\bigskip
From
\begin{equation*}
\langle{L_1}^2+{L_2}^2+{L_3}^2\rangle
=\langle{L_1}^2\rangle+\langle{L_2}^2\rangle+\langle{L_3}^2\rangle=0
\end{equation*}

and $\langle A^2\rangle\ge0$ we have
\begin{equation*}
\langle{L_1}^2\rangle=\langle{L_2}^2\rangle=\langle{L_3}^2\rangle=0
\end{equation*}

Hence the quantities represented by $L_1$, $L_2$, and $L_3$ have the definite value zero.

\end{document}
