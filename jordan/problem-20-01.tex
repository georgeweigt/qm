\input{preamble}

\fbox{\parbox{\dimexpr\linewidth-2\fboxsep-2\fboxrule}{
20-1.
Use the result of Problem 17-5 to show that
angular-momentum commutation relations
imply $\langle L_1\rangle$ and $\langle L_2\rangle$ are zero for any
state where the quantity represented by $L_3$ has a definite value.
}}

\bigskip
These are the angular-momentum commutation relations.
\begin{align*}
L_1L_2-L_2L_1&=i\hbar L_3
\\
L_2L_3-L_3L_2&=i\hbar L_1
\\
L_3L_1-L_1L_3&=i\hbar L_2
\end{align*}

By problem 17-5 we have $\langle\hbar L_1\rangle=0$
and $\langle\hbar L_2\rangle=0$ because $L_3$ has a
definite value.
Hence $\langle L_1\rangle=\langle L_2\rangle=0$.

\end{document}
