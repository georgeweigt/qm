\input{preamble}

\fbox{\parbox{\dimexpr\linewidth-2\fboxsep-2\fboxrule}{
17-5.
Suppose $A$, $B$, and $C$ represent real quantities, and
\begin{equation*}
AB-BA=iC.
\end{equation*}

Show the uncertainty-relation inequalities imply $\langle C\rangle$ is 0
for any state where the quantity represented by either $A$ or $B$ has a
definite value.
}}

\bigskip
By the inequality on p.~132 we have
\begin{align*}
\left[\langle A^2\rangle-\langle A\rangle^2\right]
\left[\langle B^2\rangle-\langle B\rangle^2\right]
&\ge\left|\tfrac{1}{2}\langle AB-BA\rangle\right|^2
\end{align*}

Substitute $iC$ for $AB-BA$ to obtain
\begin{equation*}
\left[\langle A^2\rangle-\langle A\rangle^2\right]
\left[\langle B^2\rangle-\langle B\rangle^2\right]
\ge\left|\tfrac{i}{2}\langle C\rangle\right|^2\tag{1}
\end{equation*}

Suppose $A$ has a definite value. Then the variance of $A$ is zero.
\begin{equation*}
\langle A^2\rangle-\langle A\rangle^2=0
\end{equation*}

Hence $\langle C\rangle=0$ by equation (1). Repeat for $B$.

\end{document}
