\input{preamble}

\fbox{\parbox{\dimexpr\linewidth-2\fboxsep-2\fboxrule}{
27-5.
Let
\begin{equation*}
U=1-i\varepsilon G_1-\tfrac{1}{2}\varepsilon^2{G_1}^2
\end{equation*}

and calculate $U^{-1}HU$, where
\begin{equation*}
H=\frac{1}{2m}\left({P_1}^2+{P_2}^2+{P_3}^2\right)
\end{equation*}

for a single particle. Do this two different
ways. First use this formula for $H$, the formulas
for $U^{-1}P_1U$, $U^{-1}P_2U$, $U^{-1}P_3U$, and the fact that
for each matrix $P$
\begin{equation*}
U^{-1}P^2U=U^{-1}PUU^{-1}PU=(U^{-1}PU)^2.
\end{equation*}

Then use
\begin{equation*}
U^{-1}HU=H-i\varepsilon(HG_1-G_1H)-\tfrac{1}{2}\varepsilon^2
[(HG_1-G_1H)G_1-G_1(HG_1-G_1H)],
\end{equation*}

the commutation relation
\begin{equation*}
\Omega G_1-G_1\Omega=-iK_1,
\end{equation*}

which can be written here as
\begin{equation*}
HG_1-G_1H=-iP_1,
\end{equation*}

and the commutation relation
\begin{equation*}
P_1G_1-G_1P_1=-im,
\end{equation*}

which corresponds to the way $P_1$ is changed
by Galilei transformations. If you get the same
answer, it shows the commutation relation of
$\Omega$ and $G_1$ does correspond to the way energy is
changed by a Galilei transformation.
}}

\bigskip
We have
\begin{equation*}
U^{-1}P_1U=P_1-\varepsilon m,\quad
U^{-1}P_2U=P_2,\quad
U^{-1}P_3U=P_3
\tag{1}
\end{equation*}

Hence
\begin{equation*}
U^{-1}HU
=\frac{1}{2m}\left[(P_1-\varepsilon m)^2+{P_2}^2+{P_3}^2\right]
=H-\varepsilon P_1+\tfrac{1}{2}\varepsilon^2m
\tag{2}
\end{equation*}

By the second method
\begin{align*}
U^{-1}HU&=H-i\varepsilon(HG_1-G_1H)-\tfrac{1}{2}\varepsilon^2
[(HG_1-G_1H)G_1-G_1(HG_1-G_1H)]
\\
&=H-i\varepsilon(-iP_1)-\tfrac{1}{2}\varepsilon^2
[(-iP_1)G_1-G_1(-iP_1)]
\\
&=H-\varepsilon P_1+\tfrac{1}{2}\varepsilon^2m
\end{align*}

\end{document}
