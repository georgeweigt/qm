\input{preamble}

\fbox{\parbox{\dimexpr\linewidth-2\fboxsep-2\fboxrule}{
22-2.
Show that
\begin{multline*}
(Q_1P_1+P_1Q_1+Q_2P_2+P_2Q_2+Q_3P_3+P_3Q_3)({P_1}^2+{P_2}^2+{P_3}^2)
\\
{}-({P_1}^2+{P_2}^2+{P_3}^2)(Q_1P_1+P_1Q_1+Q_2P_2+P_2Q_2+Q_3P_3+P_3Q_3)
\\
{}=i\hbar4({P_1}^2+{P_2}^2+{P_3}^2).
\end{multline*}

Use the facts that $R^{-1}$ commutes with $Q_1$, $Q_2$, $Q_3$ and that
\begin{equation*}
P_jR^{-1}-R^{-1}P_j=i\hbar Q_j(R^3)^{-1}
\end{equation*}

for $j=1,2,3$ to show that
\begin{multline*}
(Q_1P_1+P_1Q_1+Q_2P_2+P_2Q_2+Q_3P_3+P_3Q_3)R^{-1}
\\
{}-R^{-1}(Q_1P_1+P_1Q_1+Q_2P_2+P_2Q_2+Q_3P_3+P_3Q_3)
=i\hbar2R^{-1}.
\end{multline*}

Use these to show that
\begin{multline*}
(Q_1P_1+P_1Q_1+Q_2P_2+P_2Q_2+Q_3P_3+P_3Q_3)H
\\
{}-H(Q_1P_1+P_1Q_1+Q_2P_2+P_2Q_2+Q_3P_3+P_3Q_3)
\\
{}=i\hbar2\left[\frac{1}{m}({P_1}^2+{P_2}^2+{P_3}^2)-Ze^2R^{-1}\right].
\end{multline*}

Use that and the result of Problem 17-5 to find
\begin{equation*}
\frac{1}{2m}\langle {P_1}^2+{P_2}^2+{P_3}^2\rangle,\quad
-Ze^2\langle R^{-1}\rangle
\end{equation*}

and
\begin{equation*}
\langle {P_1}^2+{P_2}^2+{P_3}^2\rangle,\quad
\langle R^{-1}\rangle
\end{equation*}

for each state where the energy has a value $\varepsilon_n$.
Compare the answers with $\langle R\rangle$ found in the
last problem and with the radius and momentum
of the Bohr orbit found in Problem 19-1.
The difference between $1/\langle R^{-1}\rangle$ and $\langle R\rangle$ when
$n$ is 1 shows there are substantial probabilities
distributed over a range of possible values of
the radius that is fairly large compared to $\langle R\rangle$.
}}

\bigskip
For the first part we have
\begin{align*}
(Q_jP_j){P_j}^2-{P_j}^2(Q_jP_j)
&=Q_jP_jP_jP_j-P_jP_jQ_jP_j
\\
&=Q_jP_jP_jP_j-P_j(Q_jP_j-i\hbar)P_j
\\
&=Q_jP_jP_jP_j-P_jQ_jP_jP_j+i\hbar P_jP_j
\\
&=2i\hbar P_jP_j
\end{align*}

and
\begin{align*}
(P_jQ_j){P_j}^2-{P_j}^2(P_jQ_j)
&=P_jQ_jP_jP_j-P_jP_jP_jQ_j
\\
&=P_j(i\hbar+P_jQ_j)P_j-P_jP_jP_jQ_j
\\
&=i\hbar P_jP_j+P_jP_jQ_jP_j-P_jP_jP_jQ_j
\\
&=2i\hbar P_jP_j
\end{align*}

All other cross terms vanish by commutation.
Sum for $j=1,2,3$ to obtain
\begin{multline*}
(Q_1P_1+P_1Q_1+Q_2P_2+P_2Q_2+Q_3P_3+P_3Q_3)({P_1}^2+{P_2}^2+{P_3}^2)
\\
{}-({P_1}^2+{P_2}^2+{P_3}^2)(Q_1P_1+P_1Q_1+Q_2P_2+P_2Q_2+Q_3P_3+P_3Q_3)
\\
{}=4i\hbar{P_1}^2+4i\hbar{P_2}^2+4i\hbar{P_3}^2
\end{multline*}

For the second part we have
\begin{align*}
(Q_jP_j)R^{-1}-R^{-1}(Q_jP_j)
&=Q_jP_jR^{-1}-Q_jR^{-1}P_j
\\
&=i\hbar Q_j^2(R^3)^{-1}
\end{align*}

and
\begin{align*}
(P_jQ_j)R^{-1}-R^{-1}(P_jQ_j)
&=P_jR^{-1}Q_j-R^{-1}P_jQ_j
\\
&=i\hbar Q_j^2(R^3)^{-1}
\end{align*}

Sum for $j=1,2,3$ to obtain
\begin{multline*}
(Q_1P_1+P_1Q_1+Q_2P_2+P_2Q_2+Q_3P_3+P_3Q_3)R^{-1}
\\
{}-R^{-1}(Q_1P_1+P_1Q_1+Q_2P_2+P_2Q_2+Q_3P_3+P_3Q_3)
\\
{}=2i\hbar(Q_1^2+Q_2^2+Q_3^2)(R^3)^{-1}=2i\hbar R^2(R^3)^{-1}=2i\hbar R^{-1}
\end{multline*}

For the third part we have
\begin{equation*}
H=\frac{1}{2m}\left({P_1}^2+{P_2}^2+{P_3}^2\right)-Ze^2R^{-1}
\end{equation*}

Hence the result is $1/(2m)$ times the first part plus $-Ze^2$ times the second part.
\begin{multline*}
(Q_1P_1+P_1Q_1+Q_2P_2+P_2Q_2+Q_3P_3+P_3Q_3)H
\\
{}-H(Q_1P_1+P_1Q_1+Q_2P_2+P_2Q_2+Q_3P_3+P_3Q_3)
\\
{}=\frac{1}{2m}\left(4i\hbar{P_1}^2+4i\hbar{P_2}^2+4i\hbar{P_3}^2\right)
-Ze^2\left(2i\hbar R^{-1}\right)
\end{multline*}

Factor $2i\hbar$ to obtain
\begin{multline*}
(Q_1P_1+P_1Q_1+Q_2P_2+P_2Q_2+Q_3P_3+P_3Q_3)H
\\
{}-H(Q_1P_1+P_1Q_1+Q_2P_2+P_2Q_2+Q_3P_3+P_3Q_3)
\\
{}=2i\hbar\left[\frac{1}{m}\left({P_1}^2+{P_2}^2+{P_3}^2\right)-Ze^2R^{-1}\right]
\end{multline*}

For a definite energy $\varepsilon_n$ the Hamiltonian matrix $H$ represents a definite value.
Then by problem 17-5 we have
\begin{multline*}
\langle(Q_1P_1+P_1Q_1+Q_2P_2+P_2Q_2+Q_3P_3+P_3Q_3)H
\\
{}-H(Q_1P_1+P_1Q_1+Q_2P_2+P_2Q_2+Q_3P_3+P_3Q_3)\rangle=0
\end{multline*}

Hence for a definite energy value
\begin{equation*}
\frac{1}{m}\left({P_1}^2+{P_2}^2+{P_3}^2\right)-Ze^2R^{-1}=0
\end{equation*}

For energy
\begin{equation*}
\varepsilon_n
=-\frac{m(Ze^2)^2}{2\hbar^2n^2}
=-\tfrac{1}{2}Ze^2\langle R^{-1}\rangle
\end{equation*}

we have
\begin{equation*}
\frac{1}{2m}\langle{P_1}^2+{P_2}^2+{P_3}^2\rangle=-\varepsilon_n
\end{equation*}

and
\begin{equation*}
-Ze^2\langle R^{-1}\rangle=2\varepsilon_n
\end{equation*}

We also have
\begin{equation*}
\langle{P_1}^2+{P_2}^2+{P_3}^2\rangle=\frac{m^2(Ze^2)^2}{\hbar^2n^2}
\end{equation*}

and
\begin{equation*}
\langle R^{-1}\rangle=\frac{mZe^2}{\hbar^2n^2}
\end{equation*}

From problem 19-1
\begin{equation*}
r=\frac{\hbar^2n^2}{mZe^2}
\end{equation*}

and
\begin{equation*}
p=\frac{mZe^2}{\hbar n}
\end{equation*}

It follows that
\begin{equation*}
r=\frac{1}{\langle R^{-1}\rangle}
\end{equation*}

and
\begin{equation*}
p^2=\langle{P_1}^2+{P_2}^2+{P_3}^2\rangle
\end{equation*}

From problem 22-1
\begin{equation*}
\langle R\rangle=\frac{3}{2}\frac{\hbar^2}{mZe^2}
\end{equation*}

Hence for $n=1$ we have
\begin{equation*}
\frac{1}{\langle R^{-1}\rangle}=\tfrac{2}{3}\langle R\rangle
\end{equation*}

\end{document}
