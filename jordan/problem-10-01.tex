\input{preamble}

\fbox{\parbox{\dimexpr\linewidth-2\fboxsep-2\fboxrule}{
10-1.
Find the possible values for the quantity represented by the matrix
\begin{equation*}
\tfrac{1}{2}(1+\Sigma_1).
\end{equation*}

Do this two different ways.
First, consider the matrix as a particular case of
\begin{equation*}
x_01+x_1\Sigma_1+x_2\Sigma_2+x_3\Sigma_3
\end{equation*}

for which we know the possible values.
Second, consider what the equation
\begin{equation*}
\left[\tfrac{1}{2}(1+\Sigma_1)\right]^2=\tfrac{1}{2}(1+\Sigma_1)
\end{equation*}

tells you about the possible values.
Check that you get the same answer both ways.
}}

\bigskip
For $\tfrac{1}{2}(1+\Sigma_1)$ we have
\begin{equation*}
x_0=\tfrac{1}{2},\quad r=\sqrt{\left(\tfrac{1}{2}\right)^2+0^2+0^2}=\tfrac{1}{2}
\end{equation*}

The possible values are
\begin{align*}
x_0+r&=1
\\
x_0-r&=0
\end{align*}

For the second method we have
\begin{equation*}
\tfrac{1}{2}(1+\Sigma_1)
=\begin{pmatrix}\tfrac{1}{2}&\tfrac{1}{2}\\[1ex]\tfrac{1}{2}&\tfrac{1}{2}\end{pmatrix},\quad
\left[\tfrac{1}{2}(1+\Sigma_1)\right]^2
=\begin{pmatrix}\tfrac{1}{2}&\tfrac{1}{2}\\[1ex]\tfrac{1}{2}&\tfrac{1}{2}\end{pmatrix}
\end{equation*}

The possible values of $x$ that satisfy $x=x^2$ are 0 and 1.

\end{document}
