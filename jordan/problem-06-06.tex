\input{preamble}

\fbox{\parbox{\dimexpr\linewidth-2\fboxsep-2\fboxrule}{
6-6.
The possible values of a quantity are
\begin{equation*}
x=-2,0,2.
\end{equation*}

Suppose $\langle x\rangle=0$ and $\langle(x-\langle x\rangle)^2\rangle=4$.
Find the probabilities $\rho(-2)$, $\rho(0)$, $\rho(2)$.
}}

\bigskip
We have
\begin{equation*}
\langle x\rangle=\rho(-2)(-2)+\rho(0)(0)+\rho(2)(2)=0
\end{equation*}

Hence
\begin{equation*}
\rho(-2)=\rho(2)\tag{1}
\end{equation*}

For the variance we have
\begin{equation*}
\langle(x-\langle x\rangle)^2\rangle=\rho(-2)(-2-0)^2+\rho(0)(0-0)^2+\rho(2)(2-0)^2=4\tag{2}
\end{equation*}

Substitute (1) into (2) to obtain
\begin{equation*}
2\rho(2)(4)=4
\end{equation*}

Hence
\begin{equation*}
\rho(2)=\rho(-2)=\tfrac{1}{2}
\end{equation*}

By total probability
\begin{equation*}
\rho(0)=1-\rho(2)-\rho(-2)=0
\end{equation*}

\end{document}
