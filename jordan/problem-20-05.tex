\input{preamble}

\fbox{\parbox{\dimexpr\linewidth-2\fboxsep-2\fboxrule}{
20-5.
For the case where
\begin{equation*}
J_1=\tfrac{1}{2}\Sigma_1,\quad
J_2=\tfrac{1}{2}\Sigma_2,\quad
J_3=\tfrac{1}{2}\Sigma_3,
\end{equation*}

write out explicitly the $2\times2$ matrices for
\begin{gather*}
J_+,\quad J_-,\quad J_-J_+,\quad J_+J_-,
\\[1ex]
J^2-J_3^2-J_3,\quad J^2-J_3^2+J_3
\end{gather*}

and
\begin{equation*}
(J_-)^n(J_+)^n,\quad(J_+)^n(J_-)^n
\end{equation*}

for $n=2,3,4,\ldots$
Check that your answers give
\begin{equation*}
J_-J_+=J^2-J_3^2-J_3
\end{equation*}

and
\begin{equation*}
J_+J_-=J^2-J_3^2+J_3.
\end{equation*}

Here $j$ is $\frac{1}{2}$ and $k$ can be either $-\frac{1}{2}$ or $\frac{1}{2}$.
How big can $n$ and $n'$ be?
Thus for what value of $n$ do you know the quantities represented by
$(J_-)^{n+1}(J_+)^{n+1}$ and $(J_+)^{n+1}(J_-)^{n+1}$
are zero regardless of what $k$ is?
Do the matrices you found agree with that?
}}

\bigskip
\begin{equation*}
\Sigma_1=\begin{pmatrix}0&1\\1&0\end{pmatrix},\quad
\Sigma_2=\begin{pmatrix}0&-i\\i&0\end{pmatrix},\quad
\Sigma_3=\begin{pmatrix}1&0\\0&-1\end{pmatrix}
\end{equation*}

Hence
\begin{align*}
J_+=J_1+iJ_2&=\begin{pmatrix}0&1\\0&0\end{pmatrix}
\\
J_-=J_1-iJ_2&=\begin{pmatrix}0&0\\1&0\end{pmatrix}
\\
J_-J_+&=\begin{pmatrix}0&0\\0&1\end{pmatrix}
\\
J_+J_-&=\begin{pmatrix}1&0\\0&0\end{pmatrix}
\\
J^2-J_3^2-J_3&=\begin{pmatrix}0&0\\0&1\end{pmatrix}=J_-J_+
\\
J^2-J_3^2+J_3&=\begin{pmatrix}1&0\\0&0\end{pmatrix}=J_+J_-
\\
(J_-)^2(J_+)^2&=\begin{pmatrix}0&0\\0&0\end{pmatrix}
\\
(J_+)^2(J_-)^2&=\begin{pmatrix}0&0\\0&0\end{pmatrix}
\end{align*}

For $n\ge3$
\begin{align*}
(J_-)^n(J_+)^n=(J_-)^{n-2}(J_-)^2(J_+)^2(J_+)^{n-2}&=\begin{pmatrix}0&0\\0&0\end{pmatrix}
\\
(J_+)^n(J_-)^n=(J_+)^{n-2}(J_+)^2(J_-)^2(J_-)^{n-2}&=\begin{pmatrix}0&0\\0&0\end{pmatrix}
\end{align*}

\end{document}
