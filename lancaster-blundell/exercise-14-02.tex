\input{preamble}

\fbox{\parbox{\dimexpr\linewidth-2\fboxsep-2\fboxrule}{
(14.2)
{\it A demonstration that the photon has spin-1, with
only two spin polarizations.}

\bigskip
A photon $\gamma$ propagates with momentum
$q^\mu=(|\mathbf q|,0,0,|\mathbf q|)$.
Working with a basis where the two transverse photon polarizations are
$\epsilon_{\lambda=1}^\mu(q)=(0,1,0,0)$
and $\epsilon_{\lambda=2}^\mu(q)=(0,0,1,0)$, it may be
shown, using Noether's theorem, that the operator
$\hat S^z$, whose eigenvalue is the $z$-component spin
angular momentum of the photon, obeys the commutation relation
\begin{equation*}
\big[\hat S^z,\hat a_{\mathbf q\lambda}^\dag\big]
=i\epsilon_\lambda^{\mu=1*}(q)\hat a_{\mathbf q\lambda=2}^\dag
-i\epsilon_\lambda^{\mu=2*}(q)\hat a_{\mathbf q\lambda=1}^\dag
\tag{14.36}
\end{equation*}

(i) Define creation operators for the circular polarizations via
\begin{equation*}
\begin{aligned}
\hat b_{\mathbf qR}^\dag&=-\frac{1}{\sqrt2}
\big(\hat a_{\mathbf q1}^\dag+i\hat a_{\mathbf q2}^\dag\big)
\\
\hat b_{\mathbf qL}^\dag&=\frac{1}{\sqrt2}
\big(\hat a_{\mathbf q1}^\dag-i\hat a_{\mathbf q2}^\dag\big)
\end{aligned}
\tag{14.37}
\end{equation*}

Show that
\begin{equation*}
\begin{aligned}
\big[\hat S^z,\hat b_{\mathbf qR}^\dag\big]&=\hat b_{\mathbf qR}^\dag
\\
\big[\hat S^z,\hat b_{\mathbf qL}^\dag\big]&=-\hat b_{\mathbf qL}^\dag
\end{aligned}
\tag{14.38}
\end{equation*}

(ii) Consider the operation of $S^z$ on a state
$|\gamma_{\mathbf q\lambda}\rangle=\hat b_{\mathbf q\lambda}|0\rangle$
containing a single photon propagating along $z$:
\begin{equation*}
\hat S^z|\gamma_{\mathbf q\lambda}\rangle
=\hat S^z\hat b_{\mathbf q\lambda}^\dag|0\rangle,
\quad
\lambda=R,L
\tag{14.39}
\end{equation*}

Use the results of (i) to argue that the projection of
the photon spin along its direction of propagation
must be $S^z=\pm1$.

\bigskip
{\it See Bjorken and Drell Chapter 14 for the full version of this argument.}
}}

\bigskip
(i) By hypothesis we have
\begin{align*}
\epsilon_{\lambda=1}^{\mu=0*}(q)&=0 & \epsilon_{\lambda=2}^{\mu=0*}(q)&=0
\\
\epsilon_{\lambda=1}^{\mu=1*}(q)&=1 & \epsilon_{\lambda=2}^{\mu=1*}(q)&=0
\\
\epsilon_{\lambda=1}^{\mu=2*}(q)&=0 & \epsilon_{\lambda=2}^{\mu=2*}(q)&=1
\\
\epsilon_{\lambda=1}^{\mu=3*}(q)&=0 & \epsilon_{\lambda=2}^{\mu=3*}(q)&=0
\end{align*}

Hence
\begin{align*}
\big[\hat S^z,\hat a_{\mathbf q1}^\dag\big]
&=i\epsilon_{\lambda=1}^{\mu=1*}(q)\hat a_{\mathbf q2}^\dag
-i\epsilon_{\lambda=1}^{\mu=2*}(q)\hat a_{\mathbf q1}^\dag
=i\hat a_{\mathbf q2}^\dag
\\
\big[\hat S^z,\hat a_{\mathbf q2}^\dag\big]
&=i\epsilon_{\lambda=2}^{\mu=1*}(q)\hat a_{\mathbf q2}^\dag
-i\epsilon_{\lambda=2}^{\mu=2*}(q)\hat a_{\mathbf q1}^\dag
=-i\hat a_{\mathbf q1}^\dag
\end{align*}

It follows that
\begin{align*}
\big[\hat S^z,\hat b_{\mathbf qR}^\dag\big]
&=-\frac{1}{\sqrt2}
\Big(
\big[\hat S^z,\hat a_{\mathbf q1}^\dag\big]+
i\big[\hat S^z,\hat a_{\mathbf q2}^\dag\big]
\Big)
=-\frac{1}{\sqrt2}\big(i\hat a_{\mathbf q2}^\dag+\hat a_{\mathbf q1}^\dag\big)
=\hat b_{\mathbf qR}^\dag
\\
\big[\hat S^z,\hat b_{\mathbf qL}^\dag\big]
&=\frac{1}{\sqrt2}
\Big(
\big[\hat S^z,\hat a_{\mathbf q1}^\dag\big]-
i\big[\hat S^z,\hat a_{\mathbf q2}^\dag\big]
\Big)
=\frac{1}{\sqrt2}\big(i\hat a_{\mathbf q2}^\dag-\hat a_{\mathbf q1}^\dag\big)
=-\hat b_{\mathbf qL}^\dag
\end{align*}

(ii) From part (i) we have the commutators
\begin{align*}
\big[\hat S^z,\hat b_{\mathbf qR}^\dag\big]&=\hat b_{\mathbf qR}^\dag
\\
\big[\hat S^z,\hat b_{\mathbf qL}^\dag\big]&=-\hat b_{\mathbf qL}^\dag
\end{align*}

It follows that
\begin{align*}
\hat S^z\hat b_{\mathbf qR}^\dag&=\hat b_{\mathbf qR}^\dag\hat S^z+\hat b_{\mathbf qR}^\dag
\\
\hat S^z\hat b_{\mathbf qL}^\dag&=\hat b_{\mathbf qL}^\dag\hat S^z-\hat b_{\mathbf qL}^\dag
\end{align*}

Hence we can write
\begin{align*}
\hat S^z|\gamma_{\mathbf qR}\rangle&=\hat S^z\hat b_{\mathbf qR}^\dag|0\rangle
=\big(\hat b_{\mathbf qR}^\dag\hat S^z+\hat b_{\mathbf qR}^\dag\big)|0\rangle
\\
\hat S^z|\gamma_{\mathbf qL}\rangle&=\hat S^z\hat b_{\mathbf qL}^\dag|0\rangle
=\big(\hat b_{\mathbf qL}^\dag\hat S^z-\hat b_{\mathbf qL}^\dag\big)|0\rangle
\end{align*}

Noting that $\hat S^z|0\rangle=0$ we obtain
\begin{align*}
\hat S^z|\gamma_{\mathbf qR}\rangle&=\hat b_{\mathbf qR}^\dag|0\rangle
=|\gamma_{\mathbf qR}\rangle
\\
\hat S^z|\gamma_{\mathbf qL}\rangle&=-\hat b_{\mathbf qL}^\dag|0\rangle
=-|\gamma_{\mathbf qL}\rangle
\end{align*}

Hence
\begin{equation*}
\hat S^z|\gamma_{\mathbf q\lambda}\rangle=\pm|\gamma_{\mathbf q\lambda}\rangle
\end{equation*}
where the eigenvalue is $+1$ for $\lambda=R$ and $-1$ for $\lambda=L$.

\end{document}
