\input{preamble}

\fbox{\parbox{\dimexpr\linewidth-2\fboxsep-2\fboxrule}{
(37.3)
{\it A one-line derivation of the Dirac equation.}

\bigskip
(a) Given that the left- and right-handed parts of
the Dirac spinor for a fermion at rest are identical,
explain why we may write $(\gamma^0-1)u(p^0)=0$.

\bigskip
(b) Prove that $e^{i\mathbf K\cdot\boldsymbol\phi}
\gamma^0e^{-i\mathbf K\cdot\boldsymbol\phi}=\slashed p/m$.

\bigskip
(c) Use the result in (b) to boost
\begin{equation*}
(\gamma^0-1)u(p^0)=0
\end{equation*}
and show that you recover the Dirac equation.
}}

\bigskip
(a) Consider equation (36.29).
\begin{equation*}
u(p^0)\equiv\begin{pmatrix}u_L(p^0)\\u_R(p^0)\end{pmatrix}
=\sqrt m\begin{pmatrix}\xi\\\xi\end{pmatrix}
\tag{36.29}
\end{equation*}

It follows that
\begin{equation*}
\gamma^0u(p^0)=\sqrt m\begin{pmatrix}0&I\\I&0\end{pmatrix}\begin{pmatrix}\xi\\ \xi\end{pmatrix}
=\sqrt m\begin{pmatrix}\xi\\ \xi\end{pmatrix}=u(p^0)
\end{equation*}

Hence
\begin{equation*}
(\gamma^0-1)u(p^0)=u(p^0)-u(p^0)=0
\end{equation*}

(b) Let $\exp(-i\mathbf K\cdot\boldsymbol\sigma)$ be the negative boost such that
\begin{equation*}
\exp(-i\mathbf K\cdot\boldsymbol\sigma)u(p)=u(p^0)
\tag{1}
\end{equation*}

Trivially we have
\begin{equation*}
\exp(i\mathbf K\cdot\boldsymbol\sigma)\exp(-i\mathbf K\cdot\boldsymbol\sigma)u(p)=u(p)
\tag{2}
\end{equation*}

Combine (1) and (2) with the result from part (a) to obtain
\begin{equation*}
\exp(i\mathbf K\cdot\boldsymbol\sigma)\gamma^0\exp(-i\mathbf K\cdot\boldsymbol\sigma)u(p)=u(p)
\tag{3}
\end{equation*}

Substitute (3) into the Dirac equation $\slashed pu(p)=mu(p)$ to obtain
\begin{equation*}
\slashed pu(p)=m\exp(i\mathbf K\cdot\boldsymbol\sigma)\gamma^0\exp(-i\mathbf K\cdot\boldsymbol\sigma)u(p)
\end{equation*}

Divide through by $m$ and cancel $u(p)$ to obtain
\begin{equation*}
\slashed p/m=\exp(i\mathbf K\cdot\boldsymbol\sigma)\gamma^0\exp(-i\mathbf K\cdot\boldsymbol\sigma)
\end{equation*}

(c) Boost $(\gamma^0-1)u(p^0)=0$.
\begin{equation*}
\exp(i\mathbf K\cdot\boldsymbol\phi)(\gamma^0-1)u(p^0)
=\exp(i\mathbf K\cdot\boldsymbol\phi)\gamma^0u(p^0)-u(p)=0
\tag{4}
\end{equation*}

Substitute (1) into (4) to obtain
\begin{equation*}
\exp(i\mathbf K\cdot\boldsymbol\phi)\gamma^0\exp(-i\mathbf K\cdot\boldsymbol\phi)u(p)-u(p)=0
\end{equation*}

Then by the result from part (b) we have
\begin{equation*}
m^{-1}\slashed pu(p)-u(p)=0
\end{equation*}

Multiply through by $m$ to obtain the Dirac equation.
\begin{equation*}
(\slashed p-m)u(p)=0
\end{equation*}

\end{document}
