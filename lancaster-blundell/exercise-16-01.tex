\input{preamble}

\fbox{\parbox{\dimexpr\linewidth-2\fboxsep-2\fboxrule}{
(16.1) (a) Solve the Schrodinger equation to find the wave
functions $\phi_n(x)$ for a particle in a one-dimensional
square well defined by $V(x)=0$ for $0\le x\le a$ and
$V(x)=\infty$ for $x<0$ and $x>a$.

\bigskip
(b) Show that the retarded Green's function for this
particle is given by
\begin{equation*}
G^+(n,t_2,t_1)=\theta(t_2-t_1)e^{-i\big(\frac{n^2\pi^2}{2ma^2}\big)(t_2-t_1)}
\tag{16.39}
\end{equation*}

(c) Find $G^+(n,\omega)$ for the particle.
}}

\bigskip
(a) Let $m$ be the mass of the particle.
The Schrodinger equation is
\begin{equation*}
-\frac{\hbar^2}{2m}\frac{d^2}{dx^2}\phi_n(x)+V(x)\phi_n(x)=E_n\phi_n(x)
\end{equation*}

In the region $0\le x\le a$ we have $V(x)=0$ hence we can write
\begin{equation*}
-\frac{\hbar^2}{2m}\frac{d^2}{dx^2}\phi_n(x)=E_n\phi_n(x),
\quad
0\le x\le a
\tag{1}
\end{equation*}

Equation (1) has the following well-known solution.
\begin{equation*}
\phi_n(x)=A\sin(kx)+B\cos(kx),
\quad
k=\frac{\sqrt{2mE_n}}{\hbar}
\tag{2}
\end{equation*}

For boundary conditions we have $\phi_n(0)=0$ and $\phi_n(a)=0$ because
there is no possibility of finding the particle outside the well.
The boundary condition $\phi_n(0)=0$ forces $B=0$ because $\cos(0)=1$.
The boundary condition $\phi_n(a)=0$ forces $kx$ to be a multiple of $\pi$ at $x=a$ hence
\begin{equation*}
kx=\frac{n\pi x}{a}
\end{equation*}

It follows from the definition of $k$ in (2) that
\begin{equation*}
\frac{\sqrt{2mE_n}}{\hbar}=\frac{n\pi}{a}
\end{equation*}

Solve for $E_n$ to obtain
\begin{equation*}
E_n=\frac{\hbar^2k^2}{2m}=\frac{\hbar^2}{2m}\bigg(\frac{n\pi}{a}\bigg)^2
\end{equation*}

Hence the solution to (1) is
\begin{equation*}
\phi_n(x)=A\sin\bigg(\frac{n\pi x}{a}\bigg),
\quad
E_n=\frac{\hbar^2}{2m}\bigg(\frac{n\pi}{a}\bigg)^2
\end{equation*}

Normalize the wavefunction.
(Note that $A$ can be complex hence $|A|$.)
\begin{equation*}
1=\int_0^a\phi_n^*(x)\phi_n(x)\,dx=\tfrac{1}{2}|A|^2a
\end{equation*}

Hence
\begin{equation*}
|A|=\sqrt\frac{2}{a}
\end{equation*}

(b) Let $\psi_n(x,t)$ be the following solution to the time-dependent Schrodinger equation.
\begin{equation*}
\psi_n(x,t)=\phi_n(x)\exp\bigg({-}\frac{iE_nt}{\hbar}\bigg)
\end{equation*}

We want $G^+(n,t_2,t_1)$ such that
\begin{equation*}
\psi_n(x,t_2)=G^+(n,t_2,t_1)\psi_n(x,t_1)
\end{equation*}

The $\phi_n(x)$ cancel leaving just the time-dependent exponentials.
\begin{equation*}
\exp\bigg({-}\frac{iE_nt_2}{\hbar}\bigg)=G^+(n,t_2,t_1)\exp\bigg({-}\frac{iE_nt_1}{\hbar}\bigg)
\end{equation*}

Hence
\begin{align*}
G^+(n,t_2,t_1)
&=\theta(t_2-t_1)\exp\bigg({-}\frac{iE_nt_2}{\hbar}\bigg)\exp\bigg(\frac{iE_nt_1}{\hbar}\bigg)
\\
&=\theta(t_2-t_1)\exp\bigg({-}\frac{iE_n(t_2-t_1)}{\hbar}\bigg)
\end{align*}

(c) Take the Fourier transform of $G^+(n,t,0)$.
\begin{align*}
G^+(n,\omega)&=\int_0^\infty G^+(n,t,0)\exp\bigg(\frac{i(\omega+i\epsilon)t}{\hbar}\bigg)\,dt
\\
&=\int_0^\infty\exp\bigg({-}\frac{iE_nt}{\hbar}\bigg)\exp\bigg(\frac{i(\omega+i\epsilon)t}{\hbar}\bigg)\,dt
\\
&=\frac{i\hbar}{E_n-\omega-i\epsilon}\exp\bigg({-}\frac{i(E_n-\omega-i\epsilon)t}{\hbar}\bigg)\bigg|_0^\infty
\\
&=\frac{i\hbar}{\omega-E_n+i\epsilon}
\end{align*}

\end{document}
