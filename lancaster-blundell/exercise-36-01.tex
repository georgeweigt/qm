\input{preamble}

\fbox{\parbox{\dimexpr\linewidth-2\fboxsep-2\fboxrule}{
(36.1) (a) Show that the Dirac equation can be recast in the form
\begin{equation*}
i\frac{\partial\psi}{\partial t}=\hat H_D\psi
\tag{36.33}
\end{equation*}
where $\hat H_D=\boldsymbol\alpha\cdot\hat{\mathbf p}+\beta m$
and find $\boldsymbol\alpha$ and $\beta$ in terms of the
$\gamma$ matrices.

\bigskip
(b) Evaluate $\hat H_D^2$ and show that for a Klein-Gordon
dispersion to result we must have:

(i) that the $\alpha^i$ and $\beta$ objects all anticommute with
each other; and

(ii) $(\alpha^i)^2=(\beta)^2=1$.

\bigskip
(c) Prove the following commutation relations

(i) $[\hat H,\hat L^i]=i(\hat{\mathbf p}\times\boldsymbol\alpha)^i$
where $\hat{\mathbf L}=\hat{\mathbf x}\times\hat{\mathbf p}$.

(ii) $[\hat H,\hat S^i]=-i(\hat{\mathbf p}\times\boldsymbol\alpha)^i$
where $\hat{\mathbf S}=\frac{1}{2}\boldsymbol\Sigma$ and we define
$\boldsymbol\Sigma=\frac{i}{2}\boldsymbol\gamma\times\boldsymbol\gamma$.
}}

\bigskip
(a) Consider the following form of the Dirac equation.
\begin{equation*}
i\bigg(
\gamma^0\frac{\partial}{\partial t}
+\gamma^1\frac{\partial}{\partial x}
+\gamma^2\frac{\partial}{\partial y}
+\gamma^3\frac{\partial}{\partial z}
\bigg)\psi=m\psi
\end{equation*}

Rewrite as
\begin{equation*}
i\gamma^0\frac{\partial}{\partial t}\psi
=-i\bigg(
\gamma^1\frac{\partial}{\partial x}
+\gamma^2\frac{\partial}{\partial y}
+\gamma^3\frac{\partial}{\partial z}
\bigg)\psi+m\psi
\end{equation*}

Noting that $\gamma^0\gamma^0=I$, multiply both sides by $\gamma^0$ to obtain
\begin{equation*}
i\frac{\partial}{\partial t}\psi=-i\gamma^0\bigg(
\gamma^1\frac{\partial}{\partial x}
+\gamma^2\frac{\partial}{\partial y}
+\gamma^3\frac{\partial}{\partial z}
\bigg)\psi+m\gamma^0\psi
\end{equation*}

Hence for $\hat{\mathbf p}=-i\nabla$ we have
\begin{equation*}
\boldsymbol\alpha=\gamma^0\begin{pmatrix}\gamma^1\\ \gamma^2\\ \gamma^3\end{pmatrix},
\quad
\beta=\gamma^0
\end{equation*}

(b) The dispersion relation is
\begin{equation*}
\hat H_D^2=\hat{\mathbf p}^2+m^2
\end{equation*}

Squaring $\hat H_D$ we have
\begin{align*}
\hat H_D^2
&=(\boldsymbol\alpha\cdot\hat{\mathbf p}+\beta m)(\boldsymbol\alpha\cdot\hat{\mathbf p}+\beta m)
\\
&=(\boldsymbol\alpha\cdot\hat{\mathbf p})(\boldsymbol\alpha\cdot\hat{\mathbf p})
+(\boldsymbol\alpha\cdot\hat{\mathbf p})\beta m+\beta m(\boldsymbol\alpha\cdot\hat{\mathbf p})+\beta^2m^2
\end{align*}

The middle terms must cancel, that is
\begin{equation*}
(\boldsymbol\alpha\cdot\hat{\mathbf p})\beta m+\beta m(\boldsymbol\alpha\cdot\hat{\mathbf p})=0
\end{equation*}

Hence
\begin{equation*}
\alpha^i\beta=-\beta\alpha^i
\end{equation*}

Cross terms must cancel, that is
\begin{equation*}
\bigg(
{-}i\alpha^1\frac{\partial}{\partial x}
-i\alpha^2\frac{\partial}{\partial y}
-i\alpha^3\frac{\partial}{\partial z}
\bigg)^2
=-(\alpha^1)^2\frac{\partial^2}{\partial x^2}
-(\alpha^2)^2\frac{\partial^2}{\partial y^2}
-(\alpha^3)^2\frac{\partial^2}{\partial z^2}
\end{equation*}

Hence
\begin{equation*}
\alpha^i\alpha^j=-\alpha^j\alpha^i,\quad i\ne j
\end{equation*}

With the above anticommutation relations we now have
\begin{equation*}
\hat H_D^2=-(\alpha^1)^2\frac{\partial^2}{\partial x^2}
-(\alpha^2)^2\frac{\partial^2}{\partial y^2}
-(\alpha^3)^2\frac{\partial^2}{\partial z^2}
+\beta^2m=\hat{\mathbf p}^2+m^2
\end{equation*}

Hence
\begin{equation*}
(\alpha^i)^2=I\quad\text{and}\quad\beta^2=I
\end{equation*}

(c) We have
\begin{equation*}
\hat{\mathbf L}\psi=\hat{\mathbf x}\times\hat{\mathbf p}\psi
\end{equation*}

If follows that
\begin{align*}
[\hat H_D,\hat{\mathbf L}]\psi
&=\hat H_D\hat{\mathbf L}\psi-\hat{\mathbf L}\hat H_D\psi
\\
&=(\boldsymbol\alpha\cdot\hat{\mathbf p}+\beta m)(\hat{\mathbf x}\times\hat{\mathbf p}\psi)
-\hat{\mathbf x}\times\hat{\mathbf p}(\boldsymbol\alpha\cdot\hat{\mathbf p}+\beta m)\psi
\end{align*}

The $\beta$ terms cancel because $\beta$ commutes with $\hat{\mathbf x}$ and $\hat{\mathbf p}$.
\begin{equation*}
\beta\hat{\mathbf x}\times\hat{\mathbf p}
=\hat{\mathbf x}\times\hat{\mathbf p}\beta
\end{equation*}

Hence
\begin{equation*}
[\hat H_D,\hat{\mathbf L}]\psi
=(\boldsymbol\alpha\cdot\hat{\mathbf p})(\hat{\mathbf x}\times\hat{\mathbf p}\psi)
-\hat{\mathbf x}\times\hat{\mathbf p}(\boldsymbol\alpha\cdot\hat{\mathbf p})\psi
\tag{1}
\end{equation*}

By the product rule for differentiation
\begin{equation*}
(\boldsymbol\alpha\cdot\hat{\mathbf p})(\hat{\mathbf x}\times\hat{\mathbf p}\psi)
=(\boldsymbol\alpha\cdot\hat{\mathbf p})\hat{\mathbf x}\times\hat{\mathbf p}\psi
+\hat{\mathbf x}\times(\boldsymbol\alpha\cdot\hat{\mathbf p})\hat{\mathbf p}\psi
\tag{2}
\end{equation*}

Substitute (2) into (1).
\begin{align*}
[\hat H_D,\hat{\mathbf L}]\psi
&=(\boldsymbol\alpha\cdot\hat{\mathbf p})\hat{\mathbf x}\times\hat{\mathbf p}\psi
+\hat{\mathbf x}\times(\boldsymbol\alpha\cdot\hat{\mathbf p})\hat{\mathbf p}\psi
-\hat{\mathbf x}\times\hat{\mathbf p}(\boldsymbol\alpha\cdot\hat{\mathbf p})\psi
\\
&=(\boldsymbol\alpha\cdot\hat{\mathbf p})\hat{\mathbf x}\times\hat{\mathbf p}\psi
\end{align*}

Note that
\begin{equation*}
(\boldsymbol\alpha\cdot\hat{\mathbf p})\hat{\mathbf x}
=\boldsymbol\alpha\cdot(-i\nabla\hat{\mathbf x})
=-i\boldsymbol\alpha
\end{equation*}

Hence
\begin{equation*}
[\hat H_D,\hat{\mathbf L}]\psi
=-i\boldsymbol\alpha\times\hat{\mathbf p}\psi
=i(\hat{\mathbf p}\times\boldsymbol\alpha)\psi
\end{equation*}

For part (ii) we have
\begin{equation*}
\hat{\mathbf S}\psi
=\frac{i}{4}\boldsymbol\gamma\times\boldsymbol\gamma\psi
=\frac{i}{4}\gamma^0\boldsymbol\alpha\times\gamma^0\boldsymbol\alpha\psi
=\frac{i}{4}\gamma^0\boldsymbol\alpha\gamma^0\times\boldsymbol\alpha\psi
=-\frac{i}{4}\boldsymbol\alpha\times\boldsymbol\alpha\psi
\end{equation*}

It follows that
\begin{align*}
[\hat H_D,\hat{\mathbf S}]\psi
&=\hat H_D\hat{\mathbf S}\psi-\hat{\mathbf S}\hat H_D\psi
\\
&=-\frac{i}{4}(\boldsymbol\alpha\cdot\hat{\mathbf p}+\beta m)(\boldsymbol\alpha\times\boldsymbol\alpha\psi)
+\frac{i}{4}\boldsymbol\alpha\times\boldsymbol\alpha(\boldsymbol\alpha\cdot\hat{\mathbf p}+\beta m)\psi
\end{align*}

The $\beta$ terms cancel by the following identity.
(Recall $\beta\boldsymbol\alpha=-\boldsymbol\alpha\beta$.)
\begin{equation*}
\beta\boldsymbol\alpha\times\boldsymbol\alpha
=-\boldsymbol\alpha\beta\times\boldsymbol\alpha
=\boldsymbol\alpha\times\boldsymbol\alpha\beta
\end{equation*}

Hence
\begin{equation*}
[\hat H_D,\hat{\mathbf S}]\psi
=-\frac{i}{4}(\boldsymbol\alpha\cdot\hat{\mathbf p})(\boldsymbol\alpha\times\boldsymbol\alpha\psi)
+\frac{i}{4}\boldsymbol\alpha\times\boldsymbol\alpha(\boldsymbol\alpha\cdot\hat{\mathbf p})\psi
\end{equation*}

Consider the following component forms.
\begin{equation*}
\boldsymbol\alpha\cdot\hat{\mathbf p}
=\alpha^1\hat p^1+\alpha^2\hat p^2+\alpha^3\hat p^3,
\quad
\boldsymbol\alpha\times\boldsymbol\alpha
=\begin{pmatrix}
\alpha^2\alpha^3-\alpha^3\alpha^2
\\
\alpha^3\alpha^1-\alpha^1\alpha^3
\\
\alpha^1\alpha^2-\alpha^2\alpha^1
\end{pmatrix}
=2\begin{pmatrix}
\alpha^2\alpha^3
\\
\alpha^3\alpha^1
\\
\alpha^1\alpha^2
\end{pmatrix}
\end{equation*}

It follows that
\begin{align*}
-\frac{i}{4}
(\boldsymbol\alpha\cdot\hat{\mathbf p})(\boldsymbol\alpha\times\boldsymbol\alpha\psi)
&=\frac{i}{2}\begin{pmatrix}
-\alpha^1\alpha^2\alpha^3\hat p^1-\alpha^3\hat p^2+\alpha^2\hat p^3
\\
\alpha^3\hat p^1-\alpha^2\alpha^3\alpha^1\hat p^2-\alpha^1\hat p^3
\\
-\alpha^2\hat p^1+\alpha^1\hat p^2-\alpha^3\alpha^1\alpha^2\hat p^3
\end{pmatrix}\psi
=\frac{i}{2}(\boldsymbol\alpha\times\hat{\mathbf p}-\epsilon)\psi
\\[1ex]
\frac{i}{4}
(\boldsymbol\alpha\times\boldsymbol\alpha)(\boldsymbol\alpha\cdot\hat{\mathbf p})\psi
&=\frac{i}{2}\begin{pmatrix}
\alpha^2\alpha^3\alpha^1\hat p^1-\alpha^3\hat p^2+\alpha^2\hat p^3
\\
\alpha^3\hat p^1+\alpha^3\alpha^1\alpha^2\hat p^2-\alpha^1\hat p^3
\\
-\alpha^2\hat p^1+\alpha^1\hat p^2+\alpha^1\alpha^2\alpha^3\hat p^3
\end{pmatrix}\psi
=\frac{i}{2}(\boldsymbol\alpha\times\hat{\mathbf p}+\epsilon)\psi
\end{align*}
where
\begin{equation*}
\epsilon=\begin{pmatrix}
\alpha^1\alpha^2\alpha^3\hat p^1
\\
\alpha^2\alpha^3\alpha^1\hat p^2
\\
\alpha^3\alpha^1\alpha^2\hat p^3
\end{pmatrix}
=\begin{pmatrix}
\alpha^2\alpha^3\alpha^1\hat p^1
\\
\alpha^3\alpha^1\alpha^2\hat p^2
\\
\alpha^1\alpha^2\alpha^3\hat p^3
\end{pmatrix}
\end{equation*}

Hence
\begin{equation*}
[\hat H_D,\hat{\mathbf S}]\psi
=\frac{i}{2}(\boldsymbol\alpha\times\hat{\mathbf p}-\epsilon)\psi
+\frac{i}{2}(\boldsymbol\alpha\times\hat{\mathbf p}+\epsilon)\psi
=i\boldsymbol\alpha\times\hat{\mathbf p}\psi
=-i(\hat{\mathbf p}\times\boldsymbol\alpha)\psi
\end{equation*}

\end{document}
