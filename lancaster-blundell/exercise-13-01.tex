\input{preamble}

\fbox{\parbox{\dimexpr\linewidth-2\fboxsep-2\fboxrule}{
(13.1)
(a) Show that the conserved charge in eqn 13.16
may be written
\begin{equation*}
\hat{\mathbf Q}_{N_c}=\int d^3p\,\hat{\mathbf A}_{\mathbf p}^\dag\mathbf J\hat{\mathbf A}_{\mathbf p}
\tag{13.41}
\end{equation*}
where $\hat{\mathbf A}_{\mathbf p}=(\hat a_{1\mathbf p},\hat a_{2\mathbf p},\hat a_{3\mathbf p})$
and $\mathbf J=(J_x,J_y,J_z)$ are the spin-1 angular momentum
matrices from Chapter 9.

\bigskip
(b) Use the transformations from Exercise 3.3 to
find the form of the angular momentum matrices
appropriate to express the charge as
$\hat{\mathbf Q}_{N_c}=\int d^3p\,
\hat{\mathbf B}_{\mathbf p}^\dag\mathbf J\hat{\mathbf B}_{\mathbf p}$
where $\hat{\mathbf B}_{\mathbf p}=(\hat b_{1\mathbf p},\hat b_{0\mathbf p},\hat b_{-1\mathbf p})$.
}}

\bigskip
(a) Here is equation (13.16).
\begin{equation*}
\mathbf Q_{N_c}=\int d^3x\,(\mathbf\Phi\times\partial_0\mathbf\Phi)
\quad\text{and}\quad
\hat Q_{N_c}^a=-i\int d^3p\,\varepsilon^{abc}\hat a_{b\mathbf p}^\dag\hat a_{c\mathbf p}
\tag{13.16}
\end{equation*}

Recall that $\varepsilon^{abc}$ is the Levi-Civita symbol
\begin{equation*}
\varepsilon^1=\begin{pmatrix}
0&0&0
\\
0&0&1
\\
0&-1&0
\end{pmatrix},
\quad
\varepsilon^2=\begin{pmatrix}
0&0&-1
\\
0&0&0
\\
1&0&0
\end{pmatrix},
\quad
\varepsilon^3=\begin{pmatrix}
0&1&0
\\
-1&0&0
\\
0&0&0
\end{pmatrix}
\end{equation*}

Scalars commute with operators hence $\varepsilon^{abc}$ and $\hat a_{b\mathbf p}^\dag$ can be interchanged.
\begin{equation*}
\hat Q_{N_c}^a=-i\int d^3p\,\hat a_{b\mathbf p}^\dag\varepsilon^{abc}\hat a_{c\mathbf p}
\end{equation*}

The sum over $b$ and sum over $c$ are inner products hence we can write
\begin{equation*}
\hat Q_{N_c}^a=-i\int d^3p\,\hat{\mathbf A}_{\mathbf p}^\dag\varepsilon^a\hat{\mathbf A}_{\mathbf p}
\end{equation*}

Let
\begin{equation*}
J_x=-i\varepsilon^1,
\quad
J_y=-i\varepsilon^2,
\quad
J_z=-i\varepsilon^3
\end{equation*}

Then
\begin{equation*}
\hat{\mathbf Q}_{N_c}=\begin{pmatrix}Q_{N_c}^1\\Q_{N_c}^2\\Q_{N_c}^3\end{pmatrix}
=\int d^3p\,\hat{\mathbf A}_{\mathbf p}^\dag\begin{pmatrix}J_x\\J_y\\J_z\end{pmatrix}\hat{\mathbf A}_{\mathbf p}
=\int d^3p\,\hat{\mathbf A}_{\mathbf p}^\dag\mathbf J\hat{\mathbf A}_{\mathbf p}
\end{equation*}

(b) From Exercise 3.3 we have
\begin{align*}
\hat b_{1\mathbf p}&=\frac{1}{\sqrt2}\big({-\hat a_{1\mathbf p}}+i\hat a_{2\mathbf p}\big)
\\
\hat b_{0\mathbf p}&=\hat a_{3\mathbf p}
\\
\hat b_{-1\mathbf p}&=\frac{1}{\sqrt2}\big(\hat a_{1\mathbf p}+i\hat a_{2\mathbf p}\big)
\end{align*}

We need a unitary matrix $U$ such that
\begin{equation*}
\hat{\mathbf B}_{\mathbf p}=
\begin{pmatrix}
\frac{1}{\sqrt2}\big({-\hat a_{1\mathbf p}}+i\hat a_{2\mathbf p}\big)
\\
\hat a_{3\mathbf p}
\\
\frac{1}{\sqrt2}\big(\hat a_{1\mathbf p}+i\hat a_{2\mathbf p}\big)
\end{pmatrix}
=U\begin{pmatrix}
\hat a_{1\mathbf p}
\\
\hat a_{2\mathbf p}
\\
\hat a_{3\mathbf p}
\end{pmatrix}
=U\hat{\mathbf A}_{\mathbf p}
\end{equation*}

Hence
\begin{equation*}
U=\begin{pmatrix}
-\frac{1}{\sqrt2}&\frac{i}{\sqrt2}&0
\\
0&0&1
\\
\frac{1}{\sqrt2}&\frac{i}{\sqrt2}&0
\end{pmatrix}
\end{equation*}

It follows that
\begin{equation*}
\hat{\mathbf B}_{\mathbf p}^\dag\mathbf J'\hat{\mathbf B}_{\mathbf p}
=\big(\hat{\mathbf A}_{\mathbf p}^\dag U^\dag\big)\mathbf J'\big(U\hat{\mathbf A}_{\mathbf p}\big)
=\hat{\mathbf A}_{\mathbf p}^\dag\big(U^\dag\mathbf J'U\big)\hat{\mathbf A}_{\mathbf p}
\end{equation*}

By equivalence of (13.41) with hypothesis in (b) we have
\begin{equation*}
U^\dag\mathbf J'U=\mathbf J
\end{equation*}

Then by unitarity property $U^{-1}=U^\dag$ we can write
\begin{equation*}
\mathbf J'=U\mathbf JU^\dag
\end{equation*}

Hence the components of $\mathbf J'$ are
\begin{align*}
J_x'&=UJ_xU^\dag=\frac{1}{\sqrt2}\begin{pmatrix}0&1&0\\1&0&1\\0&1&0\end{pmatrix}
\\
J_y'&=UJ_yU^\dag=\frac{i}{\sqrt2}\begin{pmatrix}0&-1&0\\1&0&-1\\0&1&0\end{pmatrix}
\\
J_z'&=UJ_zU^\dag=\begin{pmatrix}1&0&0\\0&0&0\\0&0&-1\end{pmatrix}
\end{align*}

\end{document}
